% WARNING: This is an auto-generated file. Do not modify manually.
\makeatletter
  \newcommand{\camelhyph}[1]{\@fterfirst\c@amelhyph#1\relax }
  \def\@fterfirst #1#2{#2#1}
  \def\c@amelhyph #1{%
   \ifthenelse{\equal{#1}\relax}{}{%  Do nothing if the end has been reached
     \ifnum`#1<91 \-#1\else#1\fi%     Check whether #1 is an uppercase letter,
                                %     if so, print \-#1, otherwise #1
    \expandafter\c@amelhyph%    %     insert \c@amelhyph again.
}}
\makeatother

\newcommand{\camelcase}[1]{\camelhyph{#1}\xspace}

\subsection{Services}
\label{sec:services}


The present section lists the services offered by the current version
of the MONDO platform. For each service, the available operations are
presented, along with the parameters they take, their return type and
the exceptions they can throw. Readers familiar with Thrift may notice
that some types inherit from others, which is not supported by Thrift
itself: this is implemented through the
Ecore2Thrift~\cite{ecore2thrift} model-to-text transformation
developed in WP6, which takes into account inherited fields when
transforming an \class{EClass} into a Thrift data type.

In addition to their explicitly listed exceptions, operations
requiring authentication may produce replies with HTTP 4xx status
codes if the user fails to provide valid credentials or does not have
the required permissions. In its current version, the MONDO platform
assumes that authenticated users will have the necessary permissions.

\subsubsection{CloudATL}
\label{sec:CloudATL}
The following service operations expose the capabilities of the cloud-enabled
version of the ATL transformation language which is currently under development and
will be presented in M24 in D3.3.

\begin{longtable} {|p{2cm}|p{3.5cm}|p{2.5cm}|p{6.5cm}|}
	\caption{Operation CloudATL.launch}
	\label{tab:cloudATL-launch}\\
	\cline{2-4}%
	\multicolumn{1}{c|}{} & \textbf{Name} & \textbf{Type} & \textbf{Documentation} \\\hline 
	\textbf{Operation} & \camelcase{launch} & \camelcase{string} & Invokes a cloud-based transformation in a batch non-blocking mode.
	                                                               			Returns a token that can be used to check the status of the transformation. \\\hline
	\textbf{Parameters} & \multicolumn{3}{l|}{} \\\hline
	 & transformation & \camelcase{string} & The ATL source-code of the transformation. \\\hline 
	 & source & \camelcase{ModelSpec} & The input models of the transformation. \\\hline 
	 & target & \camelcase{ModelSpec} & The target models of the transformation. \\\hline 
	\textbf{Exceptions} & \multicolumn{3}{l|}{} \\\hline
	 & \camelcase{InvalidTransformation} & \multicolumn{2}{p{8.5cm}|}{The transformation is not valid: it is unparsable or inconsistent.} \\\hline 
	 & \camelcase{InvalidModelSpec} & \multicolumn{2}{p{8.5cm}|}{The model specification is not valid: the model or the metamodels are inaccessible or invalid.} \\\hline 
\end{longtable}
\begin{longtable} {|p{2cm}|p{3.5cm}|p{2.5cm}|p{6.5cm}|}
	\caption{Operation CloudATL.getJobs}
	\label{tab:cloudATL-getJobs}\\
	\cline{2-4}%
	\multicolumn{1}{c|}{} & \textbf{Name} & \textbf{Type} & \textbf{Documentation} \\\hline 
	\textbf{Operation} & \camelcase{getJobs} & \camelcase{list<string>} & Lists the identifiers of the transformation jobs tracked by this server. \\\hline
	\textbf{Parameters} & \multicolumn{3}{l|}{None.} \\\hline
\end{longtable}
\begin{longtable} {|p{2cm}|p{3.5cm}|p{2.5cm}|p{6.5cm}|}
	\caption{Operation CloudATL.getStatus}
	\label{tab:cloudATL-getStatus}\\
	\cline{2-4}%
	\multicolumn{1}{c|}{} & \textbf{Name} & \textbf{Type} & \textbf{Documentation} \\\hline 
	\textbf{Operation} & \camelcase{getStatus} & \camelcase{TransformationStatus} & Returns the status of a previously invoked transformation. \\\hline
	\textbf{Parameters} & \multicolumn{3}{l|}{} \\\hline
	 & token & \camelcase{string} & A valid token returned by a previous call to launch(). \\\hline 
	\textbf{Exceptions} & \multicolumn{3}{l|}{} \\\hline
	 & \camelcase{TransformationTokenNotFound} & \multicolumn{2}{p{8.5cm}|}{The specified transformation token does not exist within the invokved MONDO instance.} \\\hline 
\end{longtable}
\begin{longtable} {|p{2cm}|p{3.5cm}|p{2.5cm}|p{6.5cm}|}
	\caption{Operation CloudATL.kill}
	\label{tab:cloudATL-kill}\\
	\cline{2-4}%
	\multicolumn{1}{c|}{} & \textbf{Name} & \textbf{Type} & \textbf{Documentation} \\\hline 
	\textbf{Operation} & \camelcase{kill} & \camelcase{void} & Kills a previously invoked transformation. \\\hline
	\textbf{Parameters} & \multicolumn{3}{l|}{} \\\hline
	 & token & \camelcase{string} & A valid token returned by a previous call to launch(). \\\hline 
	\textbf{Exceptions} & \multicolumn{3}{l|}{} \\\hline
	 & \camelcase{TransformationTokenNotFound} & \multicolumn{2}{p{8.5cm}|}{The specified transformation token does not exist within the invokved MONDO instance.} \\\hline 
\end{longtable}

\subsubsection{Hawk}
\label{sec:Hawk}
The following service operations expose the capabilities of the Hawk heterogeneous model indexing
framework developed in Work Package 5. The framework is discussed in detail in D5.2 and D5.3.

\begin{longtable} {|p{2cm}|p{3.5cm}|p{2.5cm}|p{6.5cm}|}
	\caption{Operation Hawk.createInstance}
	\label{tab:hawk-createInstance}\\
	\cline{2-4}%
	\multicolumn{1}{c|}{} & \textbf{Name} & \textbf{Type} & \textbf{Documentation} \\\hline 
	\textbf{Operation} & \camelcase{createInstance} & \camelcase{void} & Creates a new Hawk instance (stopped). \\\hline
	\textbf{Parameters} & \multicolumn{3}{l|}{} \\\hline
	 & name & \camelcase{string} & The unique name of the new Hawk instance. \\\hline 
	 & backend & \camelcase{string} & The name of the backend to be used, as returned by listBackends(). \\\hline 
	 & minimumDelayMillis & \camelcase{i32} & Minimum delay between periodic synchronization in milliseconds. \\\hline 
	 & maximumDelayMillis & \camelcase{i32} & Maximum delay between periodic synchronization in milliseconds (0 to disable periodic synchronization). \\\hline 
\end{longtable}
\begin{longtable} {|p{2cm}|p{3.5cm}|p{2.5cm}|p{6.5cm}|}
	\caption{Operation Hawk.listBackends}
	\label{tab:hawk-listBackends}\\
	\cline{2-4}%
	\multicolumn{1}{c|}{} & \textbf{Name} & \textbf{Type} & \textbf{Documentation} \\\hline 
	\textbf{Operation} & \camelcase{listBackends} & \camelcase{list<string>} & Lists the names of the available storage backends. \\\hline
	\textbf{Parameters} & \multicolumn{3}{l|}{None.} \\\hline
\end{longtable}
\begin{longtable} {|p{2cm}|p{3.5cm}|p{2.5cm}|p{6.5cm}|}
	\caption{Operation Hawk.listInstances}
	\label{tab:hawk-listInstances}\\
	\cline{2-4}%
	\multicolumn{1}{c|}{} & \textbf{Name} & \textbf{Type} & \textbf{Documentation} \\\hline 
	\textbf{Operation} & \camelcase{listInstances} & \camelcase{list<HawkInstance>} & Lists the details of all Hawk instances. \\\hline
	\textbf{Parameters} & \multicolumn{3}{l|}{None.} \\\hline
\end{longtable}
\begin{longtable} {|p{2cm}|p{3.5cm}|p{2.5cm}|p{6.5cm}|}
	\caption{Operation Hawk.removeInstance}
	\label{tab:hawk-removeInstance}\\
	\cline{2-4}%
	\multicolumn{1}{c|}{} & \textbf{Name} & \textbf{Type} & \textbf{Documentation} \\\hline 
	\textbf{Operation} & \camelcase{removeInstance} & \camelcase{void} & Removes an existing Hawk instance. \\\hline
	\textbf{Parameters} & \multicolumn{3}{l|}{} \\\hline
	 & name & \camelcase{string} & The name of the Hawk instance to remove. \\\hline 
	\textbf{Exceptions} & \multicolumn{3}{l|}{} \\\hline
	 & \camelcase{HawkInstanceNotFound} & \multicolumn{2}{p{8.5cm}|}{No Hawk instance exists with that name.} \\\hline 
\end{longtable}
\begin{longtable} {|p{2cm}|p{3.5cm}|p{2.5cm}|p{6.5cm}|}
	\caption{Operation Hawk.startInstance}
	\label{tab:hawk-startInstance}\\
	\cline{2-4}%
	\multicolumn{1}{c|}{} & \textbf{Name} & \textbf{Type} & \textbf{Documentation} \\\hline 
	\textbf{Operation} & \camelcase{startInstance} & \camelcase{void} & Starts a stopped Hawk instance. \\\hline
	\textbf{Parameters} & \multicolumn{3}{l|}{} \\\hline
	 & name & \camelcase{string} & The name of the Hawk instance to start. \\\hline 
	\textbf{Exceptions} & \multicolumn{3}{l|}{} \\\hline
	 & \camelcase{HawkInstanceNotFound} & \multicolumn{2}{p{8.5cm}|}{No Hawk instance exists with that name.} \\\hline 
\end{longtable}
\begin{longtable} {|p{2cm}|p{3.5cm}|p{2.5cm}|p{6.5cm}|}
	\caption{Operation Hawk.stopInstance}
	\label{tab:hawk-stopInstance}\\
	\cline{2-4}%
	\multicolumn{1}{c|}{} & \textbf{Name} & \textbf{Type} & \textbf{Documentation} \\\hline 
	\textbf{Operation} & \camelcase{stopInstance} & \camelcase{void} & Stops a running Hawk instance. \\\hline
	\textbf{Parameters} & \multicolumn{3}{l|}{} \\\hline
	 & name & \camelcase{string} & The name of the Hawk instance to stop. \\\hline 
	\textbf{Exceptions} & \multicolumn{3}{l|}{} \\\hline
	 & \camelcase{HawkInstanceNotFound} & \multicolumn{2}{p{8.5cm}|}{No Hawk instance exists with that name.} \\\hline 
	 & \camelcase{HawkInstanceNotRunning} & \multicolumn{2}{p{8.5cm}|}{The selected Hawk instance is not running.} \\\hline 
\end{longtable}
\begin{longtable} {|p{2cm}|p{3.5cm}|p{2.5cm}|p{6.5cm}|}
	\caption{Operation Hawk.syncInstance}
	\label{tab:hawk-syncInstance}\\
	\cline{2-4}%
	\multicolumn{1}{c|}{} & \textbf{Name} & \textbf{Type} & \textbf{Documentation} \\\hline 
	\textbf{Operation} & \camelcase{syncInstance} & \camelcase{void} & Forces an immediate synchronization on a Hawk instance. \\\hline
	\textbf{Parameters} & \multicolumn{3}{l|}{} \\\hline
	 & name & \camelcase{string} & The name of the Hawk instance to stop. \\\hline 
	\textbf{Exceptions} & \multicolumn{3}{l|}{} \\\hline
	 & \camelcase{HawkInstanceNotFound} & \multicolumn{2}{p{8.5cm}|}{No Hawk instance exists with that name.} \\\hline 
	 & \camelcase{HawkInstanceNotRunning} & \multicolumn{2}{p{8.5cm}|}{The selected Hawk instance is not running.} \\\hline 
\end{longtable}
\begin{longtable} {|p{2cm}|p{3.5cm}|p{2.5cm}|p{6.5cm}|}
	\caption{Operation Hawk.registerMetamodels}
	\label{tab:hawk-registerMetamodels}\\
	\cline{2-4}%
	\multicolumn{1}{c|}{} & \textbf{Name} & \textbf{Type} & \textbf{Documentation} \\\hline 
	\textbf{Operation} & \camelcase{registerMetamodels} & \camelcase{void} & Registers a set of file-based metamodels with a Hawk instance. \\\hline
	\textbf{Parameters} & \multicolumn{3}{l|}{} \\\hline
	 & name & \camelcase{string} & The name of the Hawk instance. \\\hline 
	 & metamodel & \camelcase{list<File>} & The metamodels to register.
	                                        			More than one metamodel file can be provided in one
	                                        			request, to accomodate fragmented metamodels. \\\hline 
	\textbf{Exceptions} & \multicolumn{3}{l|}{} \\\hline
	 & \camelcase{HawkInstanceNotFound} & \multicolumn{2}{p{8.5cm}|}{No Hawk instance exists with that name.} \\\hline 
	 & \camelcase{InvalidMetamodel} & \multicolumn{2}{p{8.5cm}|}{The provided metamodel is not valid (e.g. unparsable or inconsistent).} \\\hline 
	 & \camelcase{HawkInstanceNotRunning} & \multicolumn{2}{p{8.5cm}|}{The selected Hawk instance is not running.} \\\hline 
\end{longtable}
\begin{longtable} {|p{2cm}|p{3.5cm}|p{2.5cm}|p{6.5cm}|}
	\caption{Operation Hawk.unregisterMetamodels}
	\label{tab:hawk-unregisterMetamodels}\\
	\cline{2-4}%
	\multicolumn{1}{c|}{} & \textbf{Name} & \textbf{Type} & \textbf{Documentation} \\\hline 
	\textbf{Operation} & \camelcase{unregisterMetamodels} & \camelcase{void} & Unregisters a metamodel from a Hawk instance. \\\hline
	\textbf{Parameters} & \multicolumn{3}{l|}{} \\\hline
	 & name & \camelcase{string} & The name of the Hawk instance. \\\hline 
	 & metamodel & \camelcase{list<string>} & The URIs of the metamodels. \\\hline 
	\textbf{Exceptions} & \multicolumn{3}{l|}{} \\\hline
	 & \camelcase{HawkInstanceNotFound} & \multicolumn{2}{p{8.5cm}|}{No Hawk instance exists with that name.} \\\hline 
	 & \camelcase{HawkInstanceNotRunning} & \multicolumn{2}{p{8.5cm}|}{The selected Hawk instance is not running.} \\\hline 
\end{longtable}
\begin{longtable} {|p{2cm}|p{3.5cm}|p{2.5cm}|p{6.5cm}|}
	\caption{Operation Hawk.listMetamodels}
	\label{tab:hawk-listMetamodels}\\
	\cline{2-4}%
	\multicolumn{1}{c|}{} & \textbf{Name} & \textbf{Type} & \textbf{Documentation} \\\hline 
	\textbf{Operation} & \camelcase{listMetamodels} & \camelcase{list<string>} & Lists the URIs of the registered metamodels of a Hawk instance. \\\hline
	\textbf{Parameters} & \multicolumn{3}{l|}{} \\\hline
	 & name & \camelcase{string} & The name of the Hawk instance. \\\hline 
	\textbf{Exceptions} & \multicolumn{3}{l|}{} \\\hline
	 & \camelcase{HawkInstanceNotFound} & \multicolumn{2}{p{8.5cm}|}{No Hawk instance exists with that name.} \\\hline 
	 & \camelcase{HawkInstanceNotRunning} & \multicolumn{2}{p{8.5cm}|}{The selected Hawk instance is not running.} \\\hline 
\end{longtable}
\begin{longtable} {|p{2cm}|p{3.5cm}|p{2.5cm}|p{6.5cm}|}
	\caption{Operation Hawk.listQueryLanguages}
	\label{tab:hawk-listQueryLanguages}\\
	\cline{2-4}%
	\multicolumn{1}{c|}{} & \textbf{Name} & \textbf{Type} & \textbf{Documentation} \\\hline 
	\textbf{Operation} & \camelcase{listQueryLanguages} & \camelcase{list<string>} & Lists the supported query languages and their status. \\\hline
	\textbf{Parameters} & \multicolumn{3}{l|}{} \\\hline
	 & name & \camelcase{string} & The name of the Hawk instance. \\\hline 
\end{longtable}
\begin{longtable} {|p{2cm}|p{3.5cm}|p{2.5cm}|p{6.5cm}|}
	\caption{Operation Hawk.query}
	\label{tab:hawk-query}\\
	\cline{2-4}%
	\multicolumn{1}{c|}{} & \textbf{Name} & \textbf{Type} & \textbf{Documentation} \\\hline 
	\textbf{Operation} & \camelcase{query} & \camelcase{list<QueryResult>} & Runs a query on a Hawk instance and returns a sequence of scalar values and/or model elements. \\\hline
	\textbf{Parameters} & \multicolumn{3}{l|}{} \\\hline
	 & name & \camelcase{string} & The name of the Hawk instance. \\\hline 
	 & query & \camelcase{string} & The query to be executed. \\\hline 
	 & language & \camelcase{string} & The name of the query language used (e.g. EOL, OCL). \\\hline 
	 & options & \camelcase{HawkQueryOptions} & Options for the query. \\\hline 
	\textbf{Exceptions} & \multicolumn{3}{l|}{} \\\hline
	 & \camelcase{HawkInstanceNotFound} & \multicolumn{2}{p{8.5cm}|}{No Hawk instance exists with that name.} \\\hline 
	 & \camelcase{HawkInstanceNotRunning} & \multicolumn{2}{p{8.5cm}|}{The selected Hawk instance is not running.} \\\hline 
	 & \camelcase{UnknownQueryLanguage} & \multicolumn{2}{p{8.5cm}|}{The specified query language is not supported by the operation.} \\\hline 
	 & \camelcase{InvalidQuery} & \multicolumn{2}{p{8.5cm}|}{The specified query is not valid.} \\\hline 
	 & \camelcase{FailedQuery} & \multicolumn{2}{p{8.5cm}|}{The specified query failed to complete its execution.} \\\hline 
\end{longtable}
\begin{longtable} {|p{2cm}|p{3.5cm}|p{2.5cm}|p{6.5cm}|}
	\caption{Operation Hawk.resolveProxies}
	\label{tab:hawk-resolveProxies}\\
	\cline{2-4}%
	\multicolumn{1}{c|}{} & \textbf{Name} & \textbf{Type} & \textbf{Documentation} \\\hline 
	\textbf{Operation} & \camelcase{resolveProxies} & \camelcase{list<ModelElement>} & Returns populated model elements for the provided proxies. \\\hline
	\textbf{Parameters} & \multicolumn{3}{l|}{} \\\hline
	 & name & \camelcase{string} & The name of the Hawk instance. \\\hline 
	 & ids & \camelcase{list<string>} & Proxy model element IDs to be resolved. \\\hline 
	 & includeAttributes  (optional)& \camelcase{bool} & Whether to include attributes (true) or not (false). \\\hline 
	 & includeReferences  (optional)& \camelcase{bool} & Whether to include references (true) or not (false). \\\hline 
	\textbf{Exceptions} & \multicolumn{3}{l|}{} \\\hline
	 & \camelcase{HawkInstanceNotFound} & \multicolumn{2}{p{8.5cm}|}{No Hawk instance exists with that name.} \\\hline 
	 & \camelcase{HawkInstanceNotRunning} & \multicolumn{2}{p{8.5cm}|}{The selected Hawk instance is not running.} \\\hline 
\end{longtable}
\begin{longtable} {|p{2cm}|p{3.5cm}|p{2.5cm}|p{6.5cm}|}
	\caption{Operation Hawk.addRepository}
	\label{tab:hawk-addRepository}\\
	\cline{2-4}%
	\multicolumn{1}{c|}{} & \textbf{Name} & \textbf{Type} & \textbf{Documentation} \\\hline 
	\textbf{Operation} & \camelcase{addRepository} & \camelcase{void} & Asks a Hawk instance to start monitoring a repository. \\\hline
	\textbf{Parameters} & \multicolumn{3}{l|}{} \\\hline
	 & name & \camelcase{string} & The name of the Hawk instance. \\\hline 
	 & repo & \camelcase{Repository} & The repository to monitor. \\\hline 
	 & credentials  (optional)& \camelcase{Credentials} & A valid set of credentials that has read-access to the repository. \\\hline 
	\textbf{Exceptions} & \multicolumn{3}{l|}{} \\\hline
	 & \camelcase{HawkInstanceNotFound} & \multicolumn{2}{p{8.5cm}|}{No Hawk instance exists with that name.} \\\hline 
	 & \camelcase{HawkInstanceNotRunning} & \multicolumn{2}{p{8.5cm}|}{The selected Hawk instance is not running.} \\\hline 
	 & \camelcase{UnknownRepositoryType} & \multicolumn{2}{p{8.5cm}|}{The specified repository type is not supported by the operation.} \\\hline 
	 & \camelcase{VCSAuthenticationFailed} & \multicolumn{2}{p{8.5cm}|}{The client failed to prove its identity in the VCS.} \\\hline 
\end{longtable}
\begin{longtable} {|p{2cm}|p{3.5cm}|p{2.5cm}|p{6.5cm}|}
	\caption{Operation Hawk.removeRepository}
	\label{tab:hawk-removeRepository}\\
	\cline{2-4}%
	\multicolumn{1}{c|}{} & \textbf{Name} & \textbf{Type} & \textbf{Documentation} \\\hline 
	\textbf{Operation} & \camelcase{removeRepository} & \camelcase{void} & Asks a Hawk instance to stop monitoring a repository. \\\hline
	\textbf{Parameters} & \multicolumn{3}{l|}{} \\\hline
	 & name & \camelcase{string} & The name of the Hawk instance. \\\hline 
	 & uri & \camelcase{string} & The URI of the repository to stop monitoring. \\\hline 
	\textbf{Exceptions} & \multicolumn{3}{l|}{} \\\hline
	 & \camelcase{HawkInstanceNotFound} & \multicolumn{2}{p{8.5cm}|}{No Hawk instance exists with that name.} \\\hline 
	 & \camelcase{HawkInstanceNotRunning} & \multicolumn{2}{p{8.5cm}|}{The selected Hawk instance is not running.} \\\hline 
\end{longtable}
\begin{longtable} {|p{2cm}|p{3.5cm}|p{2.5cm}|p{6.5cm}|}
	\caption{Operation Hawk.updateRepositoryCredentials}
	\label{tab:hawk-updateRepositoryCredentials}\\
	\cline{2-4}%
	\multicolumn{1}{c|}{} & \textbf{Name} & \textbf{Type} & \textbf{Documentation} \\\hline 
	\textbf{Operation} & \camelcase{updateRepositoryCredentials} & \camelcase{void} & Changes the credentials used to monitor a repository. \\\hline
	\textbf{Parameters} & \multicolumn{3}{l|}{} \\\hline
	 & name & \camelcase{string} & The name of the Hawk instance. \\\hline 
	 & uri & \camelcase{string} & The URI of the repository to update. \\\hline 
	 & cred & \camelcase{Credentials} & The new credentials to be used. \\\hline 
	\textbf{Exceptions} & \multicolumn{3}{l|}{} \\\hline
	 & \camelcase{HawkInstanceNotFound} & \multicolumn{2}{p{8.5cm}|}{No Hawk instance exists with that name.} \\\hline 
	 & \camelcase{HawkInstanceNotRunning} & \multicolumn{2}{p{8.5cm}|}{The selected Hawk instance is not running.} \\\hline 
\end{longtable}
\begin{longtable} {|p{2cm}|p{3.5cm}|p{2.5cm}|p{6.5cm}|}
	\caption{Operation Hawk.listRepositories}
	\label{tab:hawk-listRepositories}\\
	\cline{2-4}%
	\multicolumn{1}{c|}{} & \textbf{Name} & \textbf{Type} & \textbf{Documentation} \\\hline 
	\textbf{Operation} & \camelcase{listRepositories} & \camelcase{list<Repository>} & Lists the repositories monitored by a Hawk instance. \\\hline
	\textbf{Parameters} & \multicolumn{3}{l|}{} \\\hline
	 & name & \camelcase{string} & The name of the Hawk instance. \\\hline 
	\textbf{Exceptions} & \multicolumn{3}{l|}{} \\\hline
	 & \camelcase{HawkInstanceNotFound} & \multicolumn{2}{p{8.5cm}|}{No Hawk instance exists with that name.} \\\hline 
	 & \camelcase{HawkInstanceNotRunning} & \multicolumn{2}{p{8.5cm}|}{The selected Hawk instance is not running.} \\\hline 
\end{longtable}
\begin{longtable} {|p{2cm}|p{3.5cm}|p{2.5cm}|p{6.5cm}|}
	\caption{Operation Hawk.listRepositoryTypes}
	\label{tab:hawk-listRepositoryTypes}\\
	\cline{2-4}%
	\multicolumn{1}{c|}{} & \textbf{Name} & \textbf{Type} & \textbf{Documentation} \\\hline 
	\textbf{Operation} & \camelcase{listRepositoryTypes} & \camelcase{list<string>} & Lists the available repository types in this installation. \\\hline
	\textbf{Parameters} & \multicolumn{3}{l|}{None.} \\\hline
\end{longtable}
\begin{longtable} {|p{2cm}|p{3.5cm}|p{2.5cm}|p{6.5cm}|}
	\caption{Operation Hawk.listFiles}
	\label{tab:hawk-listFiles}\\
	\cline{2-4}%
	\multicolumn{1}{c|}{} & \textbf{Name} & \textbf{Type} & \textbf{Documentation} \\\hline 
	\textbf{Operation} & \camelcase{listFiles} & \camelcase{list<string>} & Lists the paths of the files of the indexed repository. \\\hline
	\textbf{Parameters} & \multicolumn{3}{l|}{} \\\hline
	 & name & \camelcase{string} & The name of the Hawk instance. \\\hline 
	 & repository & \camelcase{list<string>} & The URI of the indexed repository. \\\hline 
	 & filePatterns & \camelcase{list<string>} & File name patterns to search for (* lists all files). \\\hline 
	\textbf{Exceptions} & \multicolumn{3}{l|}{} \\\hline
	 & \camelcase{HawkInstanceNotFound} & \multicolumn{2}{p{8.5cm}|}{No Hawk instance exists with that name.} \\\hline 
	 & \camelcase{HawkInstanceNotRunning} & \multicolumn{2}{p{8.5cm}|}{The selected Hawk instance is not running.} \\\hline 
\end{longtable}
\begin{longtable} {|p{2cm}|p{3.5cm}|p{2.5cm}|p{6.5cm}|}
	\caption{Operation Hawk.configurePolling}
	\label{tab:hawk-configurePolling}\\
	\cline{2-4}%
	\multicolumn{1}{c|}{} & \textbf{Name} & \textbf{Type} & \textbf{Documentation} \\\hline 
	\textbf{Operation} & \camelcase{configurePolling} & \camelcase{void} & Sets the base polling period and max interval of a Hawk instance. \\\hline
	\textbf{Parameters} & \multicolumn{3}{l|}{} \\\hline
	 & name & \camelcase{string} & The name of the Hawk instance. \\\hline 
	 & base & \camelcase{i32} & The base polling period (in seconds). \\\hline 
	 & max & \camelcase{i32} & The maximum polling interval (in seconds). \\\hline 
	\textbf{Exceptions} & \multicolumn{3}{l|}{} \\\hline
	 & \camelcase{HawkInstanceNotFound} & \multicolumn{2}{p{8.5cm}|}{No Hawk instance exists with that name.} \\\hline 
	 & \camelcase{HawkInstanceNotRunning} & \multicolumn{2}{p{8.5cm}|}{The selected Hawk instance is not running.} \\\hline 
	 & \camelcase{InvalidPollingConfiguration} & \multicolumn{2}{p{8.5cm}|}{The polling configuration is not valid.} \\\hline 
\end{longtable}
\begin{longtable} {|p{2cm}|p{3.5cm}|p{2.5cm}|p{6.5cm}|}
	\caption{Operation Hawk.addDerivedAttribute}
	\label{tab:hawk-addDerivedAttribute}\\
	\cline{2-4}%
	\multicolumn{1}{c|}{} & \textbf{Name} & \textbf{Type} & \textbf{Documentation} \\\hline 
	\textbf{Operation} & \camelcase{addDerivedAttribute} & \camelcase{void} & Add a new derived attribute to a Hawk instance. \\\hline
	\textbf{Parameters} & \multicolumn{3}{l|}{} \\\hline
	 & name & \camelcase{string} & The name of the Hawk instance. \\\hline 
	 & spec & \camelcase{DerivedAttributeSpec} & The details of the new derived attribute. \\\hline 
	\textbf{Exceptions} & \multicolumn{3}{l|}{} \\\hline
	 & \camelcase{HawkInstanceNotFound} & \multicolumn{2}{p{8.5cm}|}{No Hawk instance exists with that name.} \\\hline 
	 & \camelcase{HawkInstanceNotRunning} & \multicolumn{2}{p{8.5cm}|}{The selected Hawk instance is not running.} \\\hline 
	 & \camelcase{InvalidDerivedAttributeSpec} & \multicolumn{2}{p{8.5cm}|}{The derived attribute specification is not valid.} \\\hline 
\end{longtable}
\begin{longtable} {|p{2cm}|p{3.5cm}|p{2.5cm}|p{6.5cm}|}
	\caption{Operation Hawk.removeDerivedAttribute}
	\label{tab:hawk-removeDerivedAttribute}\\
	\cline{2-4}%
	\multicolumn{1}{c|}{} & \textbf{Name} & \textbf{Type} & \textbf{Documentation} \\\hline 
	\textbf{Operation} & \camelcase{removeDerivedAttribute} & \camelcase{void} & Remove a derived attribute from a Hawk instance. \\\hline
	\textbf{Parameters} & \multicolumn{3}{l|}{} \\\hline
	 & name & \camelcase{string} & The name of the Hawk instance. \\\hline 
	 & spec & \camelcase{DerivedAttributeSpec} & The details of the derived attribute to be removed.
	                                             			Only the first three fields of the spec
	                                             			need to be populated. \\\hline 
	\textbf{Exceptions} & \multicolumn{3}{l|}{} \\\hline
	 & \camelcase{HawkInstanceNotFound} & \multicolumn{2}{p{8.5cm}|}{No Hawk instance exists with that name.} \\\hline 
	 & \camelcase{HawkInstanceNotRunning} & \multicolumn{2}{p{8.5cm}|}{The selected Hawk instance is not running.} \\\hline 
\end{longtable}
\begin{longtable} {|p{2cm}|p{3.5cm}|p{2.5cm}|p{6.5cm}|}
	\caption{Operation Hawk.listDerivedAttributes}
	\label{tab:hawk-listDerivedAttributes}\\
	\cline{2-4}%
	\multicolumn{1}{c|}{} & \textbf{Name} & \textbf{Type} & \textbf{Documentation} \\\hline 
	\textbf{Operation} & \camelcase{listDerivedAttributes} & \camelcase{list<DerivedAttributeSpec>} & Lists the derived attributes of a Hawk instance. Only the first three fields of the spec are currently populated. \\\hline
	\textbf{Parameters} & \multicolumn{3}{l|}{} \\\hline
	 & name & \camelcase{string} & The name of the Hawk instance. \\\hline 
	\textbf{Exceptions} & \multicolumn{3}{l|}{} \\\hline
	 & \camelcase{HawkInstanceNotFound} & \multicolumn{2}{p{8.5cm}|}{No Hawk instance exists with that name.} \\\hline 
	 & \camelcase{HawkInstanceNotRunning} & \multicolumn{2}{p{8.5cm}|}{The selected Hawk instance is not running.} \\\hline 
\end{longtable}
\begin{longtable} {|p{2cm}|p{3.5cm}|p{2.5cm}|p{6.5cm}|}
	\caption{Operation Hawk.addIndexedAttribute}
	\label{tab:hawk-addIndexedAttribute}\\
	\cline{2-4}%
	\multicolumn{1}{c|}{} & \textbf{Name} & \textbf{Type} & \textbf{Documentation} \\\hline 
	\textbf{Operation} & \camelcase{addIndexedAttribute} & \camelcase{void} & Add a new indexed attribute to a Hawk instance. \\\hline
	\textbf{Parameters} & \multicolumn{3}{l|}{} \\\hline
	 & name & \camelcase{string} & The name of the Hawk instance. \\\hline 
	 & spec & \camelcase{IndexedAttributeSpec} & The details of the new indexed attribute. \\\hline 
	\textbf{Exceptions} & \multicolumn{3}{l|}{} \\\hline
	 & \camelcase{HawkInstanceNotFound} & \multicolumn{2}{p{8.5cm}|}{No Hawk instance exists with that name.} \\\hline 
	 & \camelcase{HawkInstanceNotRunning} & \multicolumn{2}{p{8.5cm}|}{The selected Hawk instance is not running.} \\\hline 
	 & \camelcase{InvalidIndexedAttributeSpec} & \multicolumn{2}{p{8.5cm}|}{The indexed attribute specification is not valid.} \\\hline 
\end{longtable}
\begin{longtable} {|p{2cm}|p{3.5cm}|p{2.5cm}|p{6.5cm}|}
	\caption{Operation Hawk.removeIndexedAttribute}
	\label{tab:hawk-removeIndexedAttribute}\\
	\cline{2-4}%
	\multicolumn{1}{c|}{} & \textbf{Name} & \textbf{Type} & \textbf{Documentation} \\\hline 
	\textbf{Operation} & \camelcase{removeIndexedAttribute} & \camelcase{void} & Remove a indexed attribute from a Hawk instance. \\\hline
	\textbf{Parameters} & \multicolumn{3}{l|}{} \\\hline
	 & name & \camelcase{string} & The name of the Hawk instance. \\\hline 
	 & spec & \camelcase{IndexedAttributeSpec} & The details of the indexed attribute to be removed. \\\hline 
	\textbf{Exceptions} & \multicolumn{3}{l|}{} \\\hline
	 & \camelcase{HawkInstanceNotFound} & \multicolumn{2}{p{8.5cm}|}{No Hawk instance exists with that name.} \\\hline 
	 & \camelcase{HawkInstanceNotRunning} & \multicolumn{2}{p{8.5cm}|}{The selected Hawk instance is not running.} \\\hline 
\end{longtable}
\begin{longtable} {|p{2cm}|p{3.5cm}|p{2.5cm}|p{6.5cm}|}
	\caption{Operation Hawk.listIndexedAttributes}
	\label{tab:hawk-listIndexedAttributes}\\
	\cline{2-4}%
	\multicolumn{1}{c|}{} & \textbf{Name} & \textbf{Type} & \textbf{Documentation} \\\hline 
	\textbf{Operation} & \camelcase{listIndexedAttributes} & \camelcase{list<IndexedAttributeSpec>} & Lists the indexed attributes of a Hawk instance. \\\hline
	\textbf{Parameters} & \multicolumn{3}{l|}{} \\\hline
	 & name & \camelcase{string} & The name of the Hawk instance. \\\hline 
	\textbf{Exceptions} & \multicolumn{3}{l|}{} \\\hline
	 & \camelcase{HawkInstanceNotFound} & \multicolumn{2}{p{8.5cm}|}{No Hawk instance exists with that name.} \\\hline 
	 & \camelcase{HawkInstanceNotRunning} & \multicolumn{2}{p{8.5cm}|}{The selected Hawk instance is not running.} \\\hline 
\end{longtable}
\begin{longtable} {|p{2cm}|p{3.5cm}|p{2.5cm}|p{6.5cm}|}
	\caption{Operation Hawk.getModel}
	\label{tab:hawk-getModel}\\
	\cline{2-4}%
	\multicolumn{1}{c|}{} & \textbf{Name} & \textbf{Type} & \textbf{Documentation} \\\hline 
	\textbf{Operation} & \camelcase{getModel} & \camelcase{list<ModelElement>} & Returns the contents of one or more models indexed in a Hawk instance. Cross-model references are also resolved, and contained objects are always sent. \\\hline
	\textbf{Parameters} & \multicolumn{3}{l|}{} \\\hline
	 & name & \camelcase{string} & The name of the Hawk instance. \\\hline 
	 & options & \camelcase{HawkQueryOptions} & Options to limit the contents to be sent. \\\hline 
	\textbf{Exceptions} & \multicolumn{3}{l|}{} \\\hline
	 & \camelcase{HawkInstanceNotFound} & \multicolumn{2}{p{8.5cm}|}{No Hawk instance exists with that name.} \\\hline 
	 & \camelcase{HawkInstanceNotRunning} & \multicolumn{2}{p{8.5cm}|}{The selected Hawk instance is not running.} \\\hline 
\end{longtable}
\begin{longtable} {|p{2cm}|p{3.5cm}|p{2.5cm}|p{6.5cm}|}
	\caption{Operation Hawk.getRootElements}
	\label{tab:hawk-getRootElements}\\
	\cline{2-4}%
	\multicolumn{1}{c|}{} & \textbf{Name} & \textbf{Type} & \textbf{Documentation} \\\hline 
	\textbf{Operation} & \camelcase{getRootElements} & \camelcase{list<ModelElement>} & Returns the root objects of one or more models indexed in a Hawk instance. Node IDs are always sent, and contained objects are never sent. \\\hline
	\textbf{Parameters} & \multicolumn{3}{l|}{} \\\hline
	 & name & \camelcase{string} & The name of the Hawk instance. \\\hline 
	 & options & \camelcase{HawkQueryOptions} & Options to limit the contents to be sent. \\\hline 
\end{longtable}
\begin{longtable} {|p{2cm}|p{3.5cm}|p{2.5cm}|p{6.5cm}|}
	\caption{Operation Hawk.watchStateChanges}
	\label{tab:hawk-watchStateChanges}\\
	\cline{2-4}%
	\multicolumn{1}{c|}{} & \textbf{Name} & \textbf{Type} & \textbf{Documentation} \\\hline 
	\textbf{Operation} & \camelcase{watchStateChanges} & \camelcase{Subscription} & Returns subscription details to a queue of HawkStateEvents with notifications about changes in the indexer's state. \\\hline
	\textbf{Parameters} & \multicolumn{3}{l|}{} \\\hline
	 & name & \camelcase{string} & The name of the Hawk instance. \\\hline 
	\textbf{Exceptions} & \multicolumn{3}{l|}{} \\\hline
	 & \camelcase{HawkInstanceNotFound} & \multicolumn{2}{p{8.5cm}|}{No Hawk instance exists with that name.} \\\hline 
	 & \camelcase{HawkInstanceNotRunning} & \multicolumn{2}{p{8.5cm}|}{The selected Hawk instance is not running.} \\\hline 
\end{longtable}
\begin{longtable} {|p{2cm}|p{3.5cm}|p{2.5cm}|p{6.5cm}|}
	\caption{Operation Hawk.watchModelChanges}
	\label{tab:hawk-watchModelChanges}\\
	\cline{2-4}%
	\multicolumn{1}{c|}{} & \textbf{Name} & \textbf{Type} & \textbf{Documentation} \\\hline 
	\textbf{Operation} & \camelcase{watchModelChanges} & \camelcase{Subscription} & Returns subscription details to a queue of HawkChangeEvents with notifications about changes to a set of indexed models. \\\hline
	\textbf{Parameters} & \multicolumn{3}{l|}{} \\\hline
	 & name & \camelcase{string} & The name of the Hawk instance. \\\hline 
	 & repositoryUri & \camelcase{string} & The URI of the repository in which the model is contained. \\\hline 
	 & filePath & \camelcase{list<string>} & The pattern(s) for the model file(s) in the repository. \\\hline 
	 & clientID & \camelcase{string} & Unique client ID (used as suffix for the queue name). \\\hline 
	 & durableEvents & \camelcase{SubscriptionDurability} & Durability of the subscription. \\\hline 
	\textbf{Exceptions} & \multicolumn{3}{l|}{} \\\hline
	 & \camelcase{HawkInstanceNotFound} & \multicolumn{2}{p{8.5cm}|}{No Hawk instance exists with that name.} \\\hline 
	 & \camelcase{HawkInstanceNotRunning} & \multicolumn{2}{p{8.5cm}|}{The selected Hawk instance is not running.} \\\hline 
\end{longtable}

\subsubsection{OfflineCollaboration}
\label{sec:OfflineCollaboration}
The offline collaboration tool is realized in the MONDO platform through the
MONDO Offline Collaboration Server, as mentioned in D4.4~\cite{D4.4}. It extends an
off-the-shelf version control server with ``hooks'' that enforce access control and
maintain the lens relationship between the ``gold'' repository and the ``front'' repositories.
This allows users to continue using their preferred tools for interacting with the version
control systems in their day-to-day modelling activities.

Nevertheless, managing the rules to be used by the hooks and the lens relationship requires
its own API, as this is not covered by traditional VCS protocols. The rest of the section
describes a work-in-progress API for managing these access rules: the final version will be
provided in D6.8, due in M30.

\begin{longtable} {|p{2cm}|p{3.5cm}|p{2.5cm}|p{6.5cm}|}
	\caption{Operation OfflineCollaboration.addRule}
	\label{tab:offlineCollaboration-addRule}\\
	\cline{2-4}%
	\multicolumn{1}{c|}{} & \textbf{Name} & \textbf{Type} & \textbf{Documentation} \\\hline 
	\textbf{Operation} & \camelcase{addRule} & \camelcase{void} & Adds an access control rule to the specified version control system. \\\hline
	\textbf{Parameters} & \multicolumn{3}{l|}{} \\\hline
	 & repoURL & \camelcase{string} & URL of the version control system. \\\hline 
	 & rule & \camelcase{CollaborationRule} & Specification of the access control rule. \\\hline 
\end{longtable}
\begin{longtable} {|p{2cm}|p{3.5cm}|p{2.5cm}|p{6.5cm}|}
	\caption{Operation OfflineCollaboration.removeRule}
	\label{tab:offlineCollaboration-removeRule}\\
	\cline{2-4}%
	\multicolumn{1}{c|}{} & \textbf{Name} & \textbf{Type} & \textbf{Documentation} \\\hline 
	\textbf{Operation} & \camelcase{removeRule} & \camelcase{void} & Removes an access control rule from the specified version control system. \\\hline
	\textbf{Parameters} & \multicolumn{3}{l|}{} \\\hline
	 & repoURL & \camelcase{string} & URL of the version control system. \\\hline 
	 & ruleName & \camelcase{string} & Specification of the access control rule. \\\hline 
	\textbf{Exceptions} & \multicolumn{3}{l|}{} \\\hline
	 & \camelcase{CollaborationRuleNotFound} & \multicolumn{2}{p{8.5cm}|}{No collaboration rule with that name was found for the specified repository.} \\\hline 
\end{longtable}
\begin{longtable} {|p{2cm}|p{3.5cm}|p{2.5cm}|p{6.5cm}|}
	\caption{Operation OfflineCollaboration.updateRule}
	\label{tab:offlineCollaboration-updateRule}\\
	\cline{2-4}%
	\multicolumn{1}{c|}{} & \textbf{Name} & \textbf{Type} & \textbf{Documentation} \\\hline 
	\textbf{Operation} & \camelcase{updateRule} & \camelcase{void} & Updates an access control rule for the specified version control system. \\\hline
	\textbf{Parameters} & \multicolumn{3}{l|}{} \\\hline
	 & repoURL & \camelcase{string} & URL of the version control system. \\\hline 
	 & rule & \camelcase{CollaborationRule} & Specification of the access control rule. \\\hline 
	\textbf{Exceptions} & \multicolumn{3}{l|}{} \\\hline
	 & \camelcase{CollaborationRuleNotFound} & \multicolumn{2}{p{8.5cm}|}{No collaboration rule with that name was found for the specified repository.} \\\hline 
\end{longtable}
\begin{longtable} {|p{2cm}|p{3.5cm}|p{2.5cm}|p{6.5cm}|}
	\caption{Operation OfflineCollaboration.listRules}
	\label{tab:offlineCollaboration-listRules}\\
	\cline{2-4}%
	\multicolumn{1}{c|}{} & \textbf{Name} & \textbf{Type} & \textbf{Documentation} \\\hline 
	\textbf{Operation} & \camelcase{listRules} & \camelcase{list<CollaborationRule>} & Lists the access control rules for the specified version control system. \\\hline
	\textbf{Parameters} & \multicolumn{3}{l|}{} \\\hline
	 & repoURL & \camelcase{string} & URL of the version control system. \\\hline 
\end{longtable}
\begin{longtable} {|p{2cm}|p{3.5cm}|p{2.5cm}|p{6.5cm}|}
	\caption{Operation OfflineCollaboration.regenerateFrontRepositories}
	\label{tab:offlineCollaboration-regenerateFrontRepositories}\\
	\cline{2-4}%
	\multicolumn{1}{c|}{} & \textbf{Name} & \textbf{Type} & \textbf{Documentation} \\\hline 
	\textbf{Operation} & \camelcase{regenerateFrontRepositories} & \camelcase{void} & Regenerate all front repositories based on the gold repository. \\\hline
	\textbf{Parameters} & \multicolumn{3}{l|}{} \\\hline
	 & goldRepoURL & \camelcase{string} & URL of the gold repository. \\\hline 
\end{longtable}
\begin{longtable} {|p{2cm}|p{3.5cm}|p{2.5cm}|p{6.5cm}|}
	\caption{Operation OfflineCollaboration.getMyFrontRepositoryURL}
	\label{tab:offlineCollaboration-getMyFrontRepositoryURL}\\
	\cline{2-4}%
	\multicolumn{1}{c|}{} & \textbf{Name} & \textbf{Type} & \textbf{Documentation} \\\hline 
	\textbf{Operation} & \camelcase{getMyFrontRepositoryURL} & \camelcase{void} & Retrieve the front repository URL for the current user. \\\hline
	\textbf{Parameters} & \multicolumn{3}{l|}{} \\\hline
	 & goldRepoURL & \camelcase{string} & URL of the gold repository. \\\hline 
\end{longtable}

\subsubsection{Users}
\label{sec:Users}
The majority of service operations provided by the MONDO
		platform require user authentication (indicated in the top-left
		cell of each operation table) to prevent unaccountable use.
		As such, the platform needs to provide basic user management service operations
		for creating, updating and deleting user accounts.

\begin{longtable} {|p{2cm}|p{3.5cm}|p{2.5cm}|p{6.5cm}|}
	\caption{Operation Users.createUser}
	\label{tab:users-createUser}\\
	\cline{2-4}%
	\multicolumn{1}{c|}{} & \textbf{Name} & \textbf{Type} & \textbf{Documentation} \\\hline 
	\textbf{Operation} & \camelcase{createUser} & \camelcase{void} & Creates a new platform user. \\\hline
	\textbf{Parameters} & \multicolumn{3}{l|}{} \\\hline
	 & username & \camelcase{string} & A unique identifier for the user. \\\hline 
	 & password & \camelcase{string} & The desired password. \\\hline 
	 & profile & \camelcase{UserProfile} & The profile of the user. \\\hline 
	\textbf{Exceptions} & \multicolumn{3}{l|}{} \\\hline
	 & \camelcase{UserExists} & \multicolumn{2}{p{8.5cm}|}{The specified username already exists.} \\\hline 
\end{longtable}
\begin{longtable} {|p{2cm}|p{3.5cm}|p{2.5cm}|p{6.5cm}|}
	\caption{Operation Users.updateProfile}
	\label{tab:users-updateProfile}\\
	\cline{2-4}%
	\multicolumn{1}{c|}{} & \textbf{Name} & \textbf{Type} & \textbf{Documentation} \\\hline 
	\textbf{Operation} & \camelcase{updateProfile} & \camelcase{void} & Updates the profile of a platform user. \\\hline
	\textbf{Parameters} & \multicolumn{3}{l|}{} \\\hline
	 & username & \camelcase{string} & The name of the user to update the profile of. \\\hline 
	 & profile & \camelcase{UserProfile} & The updated profile of the user. \\\hline 
	\textbf{Exceptions} & \multicolumn{3}{l|}{} \\\hline
	 & \camelcase{UserNotFound} & \multicolumn{2}{p{8.5cm}|}{The specified username does not exist.} \\\hline 
\end{longtable}
\begin{longtable} {|p{2cm}|p{3.5cm}|p{2.5cm}|p{6.5cm}|}
	\caption{Operation Users.updatePassword}
	\label{tab:users-updatePassword}\\
	\cline{2-4}%
	\multicolumn{1}{c|}{} & \textbf{Name} & \textbf{Type} & \textbf{Documentation} \\\hline 
	\textbf{Operation} & \camelcase{updatePassword} & \camelcase{void} & Updates the password of a platform user. \\\hline
	\textbf{Parameters} & \multicolumn{3}{l|}{} \\\hline
	 & username & \camelcase{string} & The name of the user to update the profile of. \\\hline 
	 & newPassword & \camelcase{string} & New password to be set. \\\hline 
	\textbf{Exceptions} & \multicolumn{3}{l|}{} \\\hline
	 & \camelcase{UserNotFound} & \multicolumn{2}{p{8.5cm}|}{The specified username does not exist.} \\\hline 
\end{longtable}
\begin{longtable} {|p{2cm}|p{3.5cm}|p{2.5cm}|p{6.5cm}|}
	\caption{Operation Users.deleteUser}
	\label{tab:users-deleteUser}\\
	\cline{2-4}%
	\multicolumn{1}{c|}{} & \textbf{Name} & \textbf{Type} & \textbf{Documentation} \\\hline 
	\textbf{Operation} & \camelcase{deleteUser} & \camelcase{void} & Deletes a platform user. \\\hline
	\textbf{Parameters} & \multicolumn{3}{l|}{} \\\hline
	 & username & \camelcase{string} & The name of the user to delete. \\\hline 
	\textbf{Exceptions} & \multicolumn{3}{l|}{} \\\hline
	 & \camelcase{UserNotFound} & \multicolumn{2}{p{8.5cm}|}{The specified username does not exist.} \\\hline 
\end{longtable}


\subsection{Entities}
\label{sec:entities}
\subsubsection{AttributeSlot}

Represents a slot that can store the value(s) of an attribute of a model element.

\emph{Inherits from}: Slot.

\begin{longtable} {|p{4cm}|p{4cm}|p{7.25cm}|}
	\caption{Structure AttributeSlot}
	\label{tab:attributeSlot}\\
	\hline%
	\textbf{Field} & \textbf{Type} & \textbf{Documentation} \\\hline
		  \camelcase{name}  (\nohyphens{inherited}) & \camelcase{string}  & The name of the model element property the value of which is stored in this slot. \\\hline
		  \camelcase{value}  & \camelcase{SlotValue}  (optional) & Value of the slot (if set). \\\hline
   \multicolumn{3}{|p{.95\linewidth}|}{\textbf{Used in:} ModelElement} \\\hline
\end{longtable}

\subsubsection{CollaborationRule}

Specification of an access control rule for models.

\begin{longtable} {|p{4cm}|p{4cm}|p{7.25cm}|}
	\caption{Structure CollaborationRule}
	\label{tab:collaborationRule}\\
	\hline%
	\textbf{Field} & \textbf{Type} & \textbf{Documentation} \\\hline
		  \camelcase{body}  & \camelcase{string}  & Specification of the access rule in the MONDO Collaboration Policy Language from D4.2. \\\hline
		  \camelcase{name}  & \camelcase{string}  & Unique identifier of the rule within the VCS. \\\hline
   \multicolumn{3}{|p{.95\linewidth}|}{\textbf{Used in:} OfflineCollaboration.addRule, OfflineCollaboration.updateRule, OfflineCollaboration.listRules} \\\hline
\end{longtable}

\subsubsection{CommitItem}

Simplified entry within a commit of a repository.

\begin{longtable} {|p{4cm}|p{4cm}|p{7.25cm}|}
	\caption{Structure CommitItem}
	\label{tab:commitItem}\\
	\hline%
	\textbf{Field} & \textbf{Type} & \textbf{Documentation} \\\hline
		  \camelcase{path}  & \camelcase{string}  & Path within the repository, using / as separator. \\\hline
		  \camelcase{repoURL}  & \camelcase{string}  & URL of the repository. \\\hline
		  \camelcase{revision}  & \camelcase{string}  & Unique identifier of the revision of the repository. \\\hline
		  \camelcase{type}  & \camelcase{CommitItemChangeType}  & Type of change within the commit. \\\hline
   \multicolumn{3}{|p{.95\linewidth}|}{\textbf{Used in:} HawkModelElementAdditionEvent, HawkModelElementRemovalEvent, HawkAttributeUpdateEvent, HawkAttributeRemovalEvent, HawkReferenceAdditionEvent, HawkReferenceRemovalEvent, HawkFileAdditionEvent, HawkFileRemovalEvent} \\\hline
\end{longtable}

\subsubsection{ContainerSlot}

Represents a slot that can store other model elements within a model element.

\emph{Inherits from}: Slot.

\begin{longtable} {|p{4cm}|p{4cm}|p{7.25cm}|}
	\caption{Structure ContainerSlot}
	\label{tab:containerSlot}\\
	\hline%
	\textbf{Field} & \textbf{Type} & \textbf{Documentation} \\\hline
		  \camelcase{elements}  & \camelcase{list<ModelElement>}  & Contained elements for this slot. \\\hline
		  \camelcase{name}  (\nohyphens{inherited}) & \camelcase{string}  & The name of the model element property the value of which is stored in this slot. \\\hline
   \multicolumn{3}{|p{.95\linewidth}|}{\textbf{Used in:} ModelElement} \\\hline
\end{longtable}

\subsubsection{Credentials}

Credentials of the client in the target VCS.

\begin{longtable} {|p{4cm}|p{4cm}|p{7.25cm}|}
	\caption{Structure Credentials}
	\label{tab:credentials}\\
	\hline%
	\textbf{Field} & \textbf{Type} & \textbf{Documentation} \\\hline
		  \camelcase{password}  & \camelcase{string}  & Password for logging into the VCS. \\\hline
		  \camelcase{username}  & \camelcase{string}  & Username for logging into the VCS. \\\hline
   \multicolumn{3}{|p{.95\linewidth}|}{\textbf{Used in:} Hawk.addRepository, Hawk.updateRepositoryCredentials} \\\hline
\end{longtable}

\subsubsection{DerivedAttributeSpec}

Used to configure Hawk's derived attributes (discussed in D5.3).

\begin{longtable} {|p{4cm}|p{4cm}|p{7.25cm}|}
	\caption{Structure DerivedAttributeSpec}
	\label{tab:derivedAttributeSpec}\\
	\hline%
	\textbf{Field} & \textbf{Type} & \textbf{Documentation} \\\hline
		  \camelcase{attributeName}  & \camelcase{string}  & The name of the derived attribute. \\\hline
		  \camelcase{attributeType}  & \camelcase{string}  (optional) & The (primitive) type of the derived attribute. \\\hline
		  \camelcase{derivationLanguage}  & \camelcase{string}  (optional) & The language used to express the derivation logic. \\\hline
		  \camelcase{derivationLogic}  & \camelcase{string}  (optional) & An executable expression of the derivation logic in the language above. \\\hline
		  \camelcase{isMany}  & \camelcase{bool}  (optional) & The multiplicity of the derived attribute. \\\hline
		  \camelcase{isOrdered}  & \camelcase{bool}  (optional) & A flag specifying whether the order of the values of the derived attribute is significant (only makes sense when isMany=true). \\\hline
		  \camelcase{isUnique}  & \camelcase{bool}  (optional) & A flag specifying whether the the values of the derived attribute are unique (only makes sense when isMany=true). \\\hline
		  \camelcase{metamodelUri}  & \camelcase{string}  & The URI of the metamodel to which the derived attribute belongs. \\\hline
		  \camelcase{typeName}  & \camelcase{string}  & The name of the type to which the derived attribute belongs. \\\hline
   \multicolumn{3}{|p{.95\linewidth}|}{\textbf{Used in:} Hawk.addDerivedAttribute, Hawk.removeDerivedAttribute, Hawk.listDerivedAttributes} \\\hline
\end{longtable}

\subsubsection{File}

A file to be sent through the network.

\begin{longtable} {|p{4cm}|p{4cm}|p{7.25cm}|}
	\caption{Structure File}
	\label{tab:file}\\
	\hline%
	\textbf{Field} & \textbf{Type} & \textbf{Documentation} \\\hline
		  \camelcase{contents}  & \camelcase{binary}  & Sequence of bytes with the contents of the file. \\\hline
		  \camelcase{name}  & \camelcase{string}  & Name of the file. \\\hline
   \multicolumn{3}{|p{.95\linewidth}|}{\textbf{Used in:} Hawk.registerMetamodels} \\\hline
\end{longtable}

\subsubsection{HawkAttributeRemovalEvent}

Serialized form of an attribute removal event.

\begin{longtable} {|p{4cm}|p{4cm}|p{7.25cm}|}
	\caption{Structure HawkAttributeRemovalEvent}
	\label{tab:hawkAttributeRemovalEvent}\\
	\hline%
	\textbf{Field} & \textbf{Type} & \textbf{Documentation} \\\hline
		  \camelcase{attribute}  & \camelcase{string}  & Name of the attribute that was removed. \\\hline
		  \camelcase{id}  & \camelcase{string}  & Identifier of the model element that was changed. \\\hline
		  \camelcase{vcsItem}  & \camelcase{CommitItem}  & Entry within the commit that produced the changes. \\\hline
   \multicolumn{3}{|p{.95\linewidth}|}{\textbf{Used in:} HawkChangeEvent} \\\hline
\end{longtable}

\subsubsection{HawkAttributeUpdateEvent}

Serialized form of an attribute update event.

\begin{longtable} {|p{4cm}|p{4cm}|p{7.25cm}|}
	\caption{Structure HawkAttributeUpdateEvent}
	\label{tab:hawkAttributeUpdateEvent}\\
	\hline%
	\textbf{Field} & \textbf{Type} & \textbf{Documentation} \\\hline
		  \camelcase{attribute}  & \camelcase{string}  & Name of the attribute that was changed. \\\hline
		  \camelcase{id}  & \camelcase{string}  & Identifier of the model element that was changed. \\\hline
		  \camelcase{value}  & \camelcase{SlotValue}  & New value for the attribute. \\\hline
		  \camelcase{vcsItem}  & \camelcase{CommitItem}  & Entry within the commit that produced the changes. \\\hline
   \multicolumn{3}{|p{.95\linewidth}|}{\textbf{Used in:} HawkChangeEvent} \\\hline
\end{longtable}

\subsubsection{HawkChangeEvent}

Serialized form of a change in the indexed models of a Hawk instance.

\begin{longtable} {|p{4cm}|p{4cm}|p{7.25cm}|}
	\caption{Union HawkChangeEvent}
	\label{tab:hawkChangeEvent}\\
	\hline%
	\textbf{Field} & \textbf{Type} & \textbf{Documentation} \\\hline
		  \camelcase{fileAddition}  & \camelcase{HawkFileAdditionEvent}  & A file was added. \\\hline
		  \camelcase{fileRemoval}  & \camelcase{HawkFileRemovalEvent}  & A file was removed. \\\hline
		  \camelcase{modelElementAddition}  & \camelcase{HawkModelElementAdditionEvent}  & A model element was added. \\\hline
		  \camelcase{modelElementAttributeRemoval}  & \camelcase{HawkAttributeRemovalEvent}  & An attribute was removed. \\\hline
		  \camelcase{modelElementAttributeUpdate}  & \camelcase{HawkAttributeUpdateEvent}  & An attribute was updated. \\\hline
		  \camelcase{modelElementRemoval}  & \camelcase{HawkModelElementRemovalEvent}  & A model element was removed. \\\hline
		  \camelcase{referenceAddition}  & \camelcase{HawkReferenceAdditionEvent}  & A reference was added. \\\hline
		  \camelcase{referenceRemoval}  & \camelcase{HawkReferenceRemovalEvent}  & A reference was removed. \\\hline
		  \camelcase{syncEnd}  & \camelcase{HawkSynchronizationEndEvent}  & Synchronization ended. \\\hline
		  \camelcase{syncStart}  & \camelcase{HawkSynchronizationStartEvent}  & Synchronization started. \\\hline
   
\end{longtable}

\subsubsection{HawkFileAdditionEvent}

Serialized form of a file addition event.

\begin{longtable} {|p{4cm}|p{4cm}|p{7.25cm}|}
	\caption{Structure HawkFileAdditionEvent}
	\label{tab:hawkFileAdditionEvent}\\
	\hline%
	\textbf{Field} & \textbf{Type} & \textbf{Documentation} \\\hline
		  \camelcase{vcsItem}  & \camelcase{CommitItem}  & Reference to file that was added, including VCS metadata. \\\hline
   \multicolumn{3}{|p{.95\linewidth}|}{\textbf{Used in:} HawkChangeEvent} \\\hline
\end{longtable}

\subsubsection{HawkFileRemovalEvent}

A file was removed.

\begin{longtable} {|p{4cm}|p{4cm}|p{7.25cm}|}
	\caption{Structure HawkFileRemovalEvent}
	\label{tab:hawkFileRemovalEvent}\\
	\hline%
	\textbf{Field} & \textbf{Type} & \textbf{Documentation} \\\hline
		  \camelcase{vcsItem}  & \camelcase{CommitItem}  & Reference to file that was removed, including VCS metadata. \\\hline
   \multicolumn{3}{|p{.95\linewidth}|}{\textbf{Used in:} HawkChangeEvent} \\\hline
\end{longtable}

\subsubsection{HawkInstance}

Status of a Hawk instance.

\begin{longtable} {|p{4cm}|p{4cm}|p{7.25cm}|}
	\caption{Structure HawkInstance}
	\label{tab:hawkInstance}\\
	\hline%
	\textbf{Field} & \textbf{Type} & \textbf{Documentation} \\\hline
		  \camelcase{message}  & \camelcase{string}  & Last info message from the instance. \\\hline
		  \camelcase{name}  & \camelcase{string}  & The name of the instance. \\\hline
		  \camelcase{state}  & \camelcase{HawkState}  & Current state of the instance. \\\hline
   \multicolumn{3}{|p{.95\linewidth}|}{\textbf{Used in:} Hawk.listInstances} \\\hline
\end{longtable}

\subsubsection{HawkModelElementAdditionEvent}

Serialized form of a model element addition event.

\begin{longtable} {|p{4cm}|p{4cm}|p{7.25cm}|}
	\caption{Structure HawkModelElementAdditionEvent}
	\label{tab:hawkModelElementAdditionEvent}\\
	\hline%
	\textbf{Field} & \textbf{Type} & \textbf{Documentation} \\\hline
		  \camelcase{id}  & \camelcase{string}  & Identifier of the model element that was added. \\\hline
		  \camelcase{metamodelURI}  & \camelcase{string}  & Metamodel URI of the type of the model element. \\\hline
		  \camelcase{typeName}  & \camelcase{string}  & Name of the type of the model element. \\\hline
		  \camelcase{vcsItem}  & \camelcase{CommitItem}  & Entry within the commit that produced the changes. \\\hline
   \multicolumn{3}{|p{.95\linewidth}|}{\textbf{Used in:} HawkChangeEvent} \\\hline
\end{longtable}

\subsubsection{HawkModelElementRemovalEvent}

Serialized form of a model element removal event.

\begin{longtable} {|p{4cm}|p{4cm}|p{7.25cm}|}
	\caption{Structure HawkModelElementRemovalEvent}
	\label{tab:hawkModelElementRemovalEvent}\\
	\hline%
	\textbf{Field} & \textbf{Type} & \textbf{Documentation} \\\hline
		  \camelcase{id}  & \camelcase{string}  & Identifier of the model element that was removed. \\\hline
		  \camelcase{vcsItem}  & \camelcase{CommitItem}  & Entry within the commit that produced the changes. \\\hline
   \multicolumn{3}{|p{.95\linewidth}|}{\textbf{Used in:} HawkChangeEvent} \\\hline
\end{longtable}

\subsubsection{HawkQueryOptions}

Options for a Hawk query.

\begin{longtable} {|p{4cm}|p{4cm}|p{7.25cm}|}
	\caption{Structure HawkQueryOptions}
	\label{tab:hawkQueryOptions}\\
	\hline%
	\textbf{Field} & \textbf{Type} & \textbf{Documentation} \\\hline
		  \camelcase{defaultNamespaces}  & \camelcase{string}  (optional) & The default namespaces to be used to resolve ambiguous unqualified types. \\\hline
		  \camelcase{filePatterns}  & \camelcase{list<string>}  (optional) & The file patterns for the query (e.g. *.uml). \\\hline
		  \camelcase{includeAttributes}  & \camelcase{bool}  (optional) & Whether to include attributes (true) or not (false) in model element results. \\\hline
		  \camelcase{includeContained}  & \camelcase{bool}  (optional) & Whether to include all the child elements of the model element results (true) or not (false). \\\hline
		  \camelcase{includeNodeIDs}  & \camelcase{bool}  (optional) & Whether to include node IDs (true) or not (false) in model element results. \\\hline
		  \camelcase{includeReferences}  & \camelcase{bool}  (optional) & Whether to include references (true) or not (false) in model element results. \\\hline
		  \camelcase{repositoryPattern}  & \camelcase{string}  (optional) & The repository for the query (or * for all repositories). \\\hline
   \multicolumn{3}{|p{.95\linewidth}|}{\textbf{Used in:} Hawk.query, Hawk.getModel, Hawk.getRootElements} \\\hline
\end{longtable}

\subsubsection{HawkReferenceAdditionEvent}

Serialized form of a reference addition event.

\begin{longtable} {|p{4cm}|p{4cm}|p{7.25cm}|}
	\caption{Structure HawkReferenceAdditionEvent}
	\label{tab:hawkReferenceAdditionEvent}\\
	\hline%
	\textbf{Field} & \textbf{Type} & \textbf{Documentation} \\\hline
		  \camelcase{refName}  & \camelcase{string}  & Name of the reference that was added. \\\hline
		  \camelcase{sourceId}  & \camelcase{string}  & Identifier of the source model element. \\\hline
		  \camelcase{targetId}  & \camelcase{string}  & Identifier of the target model element. \\\hline
		  \camelcase{vcsItem}  & \camelcase{CommitItem}  & Entry within the commit that produced the changes. \\\hline
   \multicolumn{3}{|p{.95\linewidth}|}{\textbf{Used in:} HawkChangeEvent} \\\hline
\end{longtable}

\subsubsection{HawkReferenceRemovalEvent}

Serialized form of a reference removal event.

\begin{longtable} {|p{4cm}|p{4cm}|p{7.25cm}|}
	\caption{Structure HawkReferenceRemovalEvent}
	\label{tab:hawkReferenceRemovalEvent}\\
	\hline%
	\textbf{Field} & \textbf{Type} & \textbf{Documentation} \\\hline
		  \camelcase{refName}  & \camelcase{string}  & Name of the reference that was removed. \\\hline
		  \camelcase{sourceId}  & \camelcase{string}  & Identifier of the source model element. \\\hline
		  \camelcase{targetId}  & \camelcase{string}  & Identifier of the target model element. \\\hline
		  \camelcase{vcsItem}  & \camelcase{CommitItem}  & Entry within the commit that produced the changes. \\\hline
   \multicolumn{3}{|p{.95\linewidth}|}{\textbf{Used in:} HawkChangeEvent} \\\hline
\end{longtable}

\subsubsection{HawkStateEvent}

Serialized form of a change in the state of a Hawk instance.

\begin{longtable} {|p{4cm}|p{4cm}|p{7.25cm}|}
	\caption{Structure HawkStateEvent}
	\label{tab:hawkStateEvent}\\
	\hline%
	\textbf{Field} & \textbf{Type} & \textbf{Documentation} \\\hline
		  \camelcase{message}  & \camelcase{string}  & Short message about the current status of the server. \\\hline
		  \camelcase{state}  & \camelcase{HawkState}  & State of the Hawk instance. \\\hline
		  \camelcase{timestamp}  & \camelcase{i64}  & Timestamp for this state change. \\\hline
   
\end{longtable}

\subsubsection{HawkSynchronizationEndEvent}

Serialized form of a sync end event.

\begin{longtable} {|p{4cm}|p{4cm}|p{7.25cm}|}
	\caption{Structure HawkSynchronizationEndEvent}
	\label{tab:hawkSynchronizationEndEvent}\\
	\hline%
	\textbf{Field} & \textbf{Type} & \textbf{Documentation} \\\hline
		  \camelcase{timestampNanos}  & \camelcase{i64}  & Local timestamp, measured in nanoseconds. Only meant to be used to compute synchronization cost. \\\hline
   \multicolumn{3}{|p{.95\linewidth}|}{\textbf{Used in:} HawkChangeEvent} \\\hline
\end{longtable}

\subsubsection{HawkSynchronizationStartEvent}

Serialized form of a sync start event.

\begin{longtable} {|p{4cm}|p{4cm}|p{7.25cm}|}
	\caption{Structure HawkSynchronizationStartEvent}
	\label{tab:hawkSynchronizationStartEvent}\\
	\hline%
	\textbf{Field} & \textbf{Type} & \textbf{Documentation} \\\hline
		  \camelcase{timestampNanos}  & \camelcase{i64}  & Local timestamp, measured in nanoseconds. Only meant to be used to compute synchronization cost. \\\hline
   \multicolumn{3}{|p{.95\linewidth}|}{\textbf{Used in:} HawkChangeEvent} \\\hline
\end{longtable}

\subsubsection{IndexedAttributeSpec}

Used to configure Hawk's indexed attributes (discussed in D5.3).

\begin{longtable} {|p{4cm}|p{4cm}|p{7.25cm}|}
	\caption{Structure IndexedAttributeSpec}
	\label{tab:indexedAttributeSpec}\\
	\hline%
	\textbf{Field} & \textbf{Type} & \textbf{Documentation} \\\hline
		  \camelcase{attributeName}  & \camelcase{string}  & The name of the indexed attribute. \\\hline
		  \camelcase{metamodelUri}  & \camelcase{string}  & The URI of the metamodel to which the indexed attribute belongs. \\\hline
		  \camelcase{typeName}  & \camelcase{string}  & The name of the type to which the indexed attribute belongs. \\\hline
   \multicolumn{3}{|p{.95\linewidth}|}{\textbf{Used in:} Hawk.addIndexedAttribute, Hawk.removeIndexedAttribute, Hawk.listIndexedAttributes} \\\hline
\end{longtable}

\subsubsection{MixedReference}

Represents a reference to a model element: it can be an identifier or a position.
Only used when the same ReferenceSlot has both identifier-based and position-based references.
This may be the case if we are retrieving a subset of the model which has references
between its elements and with elements outside the subset at the same time.

\begin{longtable} {|p{4cm}|p{4cm}|p{7.25cm}|}
	\caption{Union MixedReference}
	\label{tab:mixedReference}\\
	\hline%
	\textbf{Field} & \textbf{Type} & \textbf{Documentation} \\\hline
		  \camelcase{id}  & \camelcase{string}  & Identifier-based reference to a model element. \\\hline
		  \camelcase{position}  & \camelcase{i32}  & Position-based reference to a model element. \\\hline
   \multicolumn{3}{|p{.95\linewidth}|}{\textbf{Used in:} ReferenceSlot} \\\hline
\end{longtable}

\subsubsection{ModelElement}

Represents a model element.

\begin{longtable} {|p{4cm}|p{4cm}|p{7.25cm}|}
	\caption{Structure ModelElement}
	\label{tab:modelElement}\\
	\hline%
	\textbf{Field} & \textbf{Type} & \textbf{Documentation} \\\hline
		  \camelcase{attributes}  & \camelcase{list<AttributeSlot>}  (optional) & Slots holding the values of the model element's attributes, if any have been set. \\\hline
		  \camelcase{containers}  & \camelcase{list<ContainerSlot>}  (optional) & Slots holding contained model elements, if any have been set. \\\hline
		  \camelcase{file}  & \camelcase{string}  (optional) & Name of the file to which the element belongs (not set if equal to that of the previous model element). \\\hline
		  \camelcase{id}  & \camelcase{string}  (optional) & Unique ID of the model element (not set if using position-based references). \\\hline
		  \camelcase{metamodelUri}  & \camelcase{string}  (optional) & URI of the metamodel to which the type of the element belongs (not set if equal to that of the previous model element). \\\hline
		  \camelcase{references}  & \camelcase{list<ReferenceSlot>}  (optional) & Slots holding the values of the model element's references, if any have been set. \\\hline
		  \camelcase{repositoryURL}  & \camelcase{string}  (optional) & URI of the repository to which the element belongs (not set if equal to that of the previous model element). \\\hline
		  \camelcase{typeName}  & \camelcase{string}  (optional) & Name of the type that the model element is an instance of (not set if equal to that of the previous model element). \\\hline
   \multicolumn{3}{|p{.95\linewidth}|}{\textbf{Used in:} Hawk.resolveProxies, Hawk.getModel, Hawk.getRootElements, ContainerSlot, QueryResult} \\\hline
\end{longtable}

\subsubsection{ModelElementType}

Represents a type of model element.

\begin{longtable} {|p{4cm}|p{4cm}|p{7.25cm}|}
	\caption{Structure ModelElementType}
	\label{tab:modelElementType}\\
	\hline%
	\textbf{Field} & \textbf{Type} & \textbf{Documentation} \\\hline
		  \camelcase{attributes}  & \camelcase{list<SlotMetadata>}  & Metadata for the attribute slots. \\\hline
		  \camelcase{id}  & \camelcase{string}  & Unique ID of the model element type. \\\hline
		  \camelcase{metamodelUri}  & \camelcase{string}  & URI of the metamodel to which the type belongs. \\\hline
		  \camelcase{references}  & \camelcase{list<SlotMetadata>}  & Metadata for the reference slots. \\\hline
		  \camelcase{typeName}  & \camelcase{string}  & Name of the type. \\\hline
   \multicolumn{3}{|p{.95\linewidth}|}{\textbf{Used in:} QueryResult} \\\hline
\end{longtable}

\subsubsection{ModelSpec}

Captures information about source/target models of ATL transformations.

\begin{longtable} {|p{4cm}|p{4cm}|p{7.25cm}|}
	\caption{Structure ModelSpec}
	\label{tab:modelSpec}\\
	\hline%
	\textbf{Field} & \textbf{Type} & \textbf{Documentation} \\\hline
		  \camelcase{metamodelUris}  & \camelcase{list<string>}  & The URIs of the metamodels to which elements of the model conform. \\\hline
		  \camelcase{uri}  & \camelcase{string}  & The URI from which the model will be loaded or to which it will be persisted. \\\hline
   \multicolumn{3}{|p{.95\linewidth}|}{\textbf{Used in:} CloudATL.launch, InvalidModelSpec} \\\hline
\end{longtable}

\subsubsection{QueryResult}

Union type for a scalar value or a reference to a model element. Useful for heterogeneous collections.

\emph{Inherits from}: Value.

\begin{longtable} {|p{4cm}|p{4cm}|p{7.25cm}|}
	\caption{Union QueryResult}
	\label{tab:queryResult}\\
	\hline%
	\textbf{Field} & \textbf{Type} & \textbf{Documentation} \\\hline
		  \camelcase{vBoolean}  (\nohyphens{inherited}) & \camelcase{bool}  & Boolean (true/false) value. \\\hline
		  \camelcase{vByte}  (\nohyphens{inherited}) & \camelcase{byte}  & 8-bit signed integer value. \\\hline
		  \camelcase{vDouble}  (\nohyphens{inherited}) & \camelcase{double}  & 64-bit floating point value. \\\hline
		  \camelcase{vInteger}  (\nohyphens{inherited}) & \camelcase{i32}  & 32-bit signed integer value. \\\hline
		  \camelcase{vLong}  (\nohyphens{inherited}) & \camelcase{i64}  & 64-bit signed integer value. \\\hline
		  \camelcase{vModelElement}  & \camelcase{ModelElement}  & Encoded model element. \\\hline
		  \camelcase{vModelElementType}  & \camelcase{ModelElementType}  & Encoded model element type. \\\hline
		  \camelcase{vShort}  (\nohyphens{inherited}) & \camelcase{i16}  & 16-bit signed integer value. \\\hline
		  \camelcase{vString}  (\nohyphens{inherited}) & \camelcase{string}  & Sequence of UTF8 characters. \\\hline
   \multicolumn{3}{|p{.95\linewidth}|}{\textbf{Used in:} Hawk.query} \\\hline
\end{longtable}

\subsubsection{ReferenceSlot}

Represents a slot that can store the value(s) of a reference  of a model element.
References can be expressed as positions within a result tree (using pre-order traversal)
or IDs. id, ids, position, positions and mixed are all mutually exclusive. At least one position
or one ID must be given.

\emph{Inherits from}: Slot.

\begin{longtable} {|p{4cm}|p{4cm}|p{7.25cm}|}
	\caption{Structure ReferenceSlot}
	\label{tab:referenceSlot}\\
	\hline%
	\textbf{Field} & \textbf{Type} & \textbf{Documentation} \\\hline
		  \camelcase{id}  & \camelcase{string}  (optional) & Unique identifier of the referenced element (if there is only one ID based reference in this slot). \\\hline
		  \camelcase{ids}  & \camelcase{list<string>}  (optional) & Unique identifiers of the referenced elements (if more than one). \\\hline
		  \camelcase{mixed}  & \camelcase{list<MixedReference>}  (optional) & Mix of identifier- and position-bsaed references (if there is at least one position and one ID). \\\hline
		  \camelcase{name}  (\nohyphens{inherited}) & \camelcase{string}  & The name of the model element property the value of which is stored in this slot. \\\hline
		  \camelcase{position}  & \camelcase{i32}  (optional) & Position of the referenced element (if there is only one position-based reference in this slot). \\\hline
		  \camelcase{positions}  & \camelcase{list<i32>}  (optional) & Positions of the referenced elements (if more than one). \\\hline
   \multicolumn{3}{|p{.95\linewidth}|}{\textbf{Used in:} ModelElement} \\\hline
\end{longtable}

\subsubsection{Repository}

Entity that represents a model repository.

\begin{longtable} {|p{4cm}|p{4cm}|p{7.25cm}|}
	\caption{Structure Repository}
	\label{tab:repository}\\
	\hline%
	\textbf{Field} & \textbf{Type} & \textbf{Documentation} \\\hline
		  \camelcase{type}  & \camelcase{string}  & The type of repository. \\\hline
		  \camelcase{uri}  & \camelcase{string}  & The URI to the repository. \\\hline
   \multicolumn{3}{|p{.95\linewidth}|}{\textbf{Used in:} Hawk.addRepository, Hawk.listRepositories} \\\hline
\end{longtable}

\subsubsection{Slot}

Represents a slot that can store the value(s) of a property of a model element.

\emph{Inherited by}: AttributeSlot, ReferenceSlot, ContainerSlot.

\begin{longtable} {|p{4cm}|p{4cm}|p{7.25cm}|}
	\caption{Structure Slot}
	\label{tab:slot}\\
	\hline%
	\textbf{Field} & \textbf{Type} & \textbf{Documentation} \\\hline
		  \camelcase{name}  & \camelcase{string}  & The name of the model element property the value of which is stored in this slot. \\\hline
   
\end{longtable}

\subsubsection{SlotMetadata}

Represents the metadata of a slot in a model element type.

\begin{longtable} {|p{4cm}|p{4cm}|p{7.25cm}|}
	\caption{Structure SlotMetadata}
	\label{tab:slotMetadata}\\
	\hline%
	\textbf{Field} & \textbf{Type} & \textbf{Documentation} \\\hline
		  \camelcase{isMany}  & \camelcase{bool}  & True if this slot holds a collection of values instead of a single value. \\\hline
		  \camelcase{isOrdered}  & \camelcase{bool}  & True if the values in this slot are ordered. \\\hline
		  \camelcase{isUnique}  & \camelcase{bool}  & True if the value of this slot must be unique within its containing model. \\\hline
		  \camelcase{name}  & \camelcase{string}  & The name of the model element property that is stored in this slot. \\\hline
		  \camelcase{type}  & \camelcase{string}  & The type of the values in this slot. \\\hline
   \multicolumn{3}{|p{.95\linewidth}|}{\textbf{Used in:} ModelElementType} \\\hline
\end{longtable}

\subsubsection{SlotValue}

Union type for a single scalar value or a homogeneous collection of scalar values.

\emph{Inherits from}: Value.

\begin{longtable} {|p{4cm}|p{4cm}|p{7.25cm}|}
	\caption{Union SlotValue}
	\label{tab:slotValue}\\
	\hline%
	\textbf{Field} & \textbf{Type} & \textbf{Documentation} \\\hline
		  \camelcase{vBoolean}  (\nohyphens{inherited}) & \camelcase{bool}  & Boolean (true/false) value. \\\hline
		  \camelcase{vBooleans}  & \camelcase{list<bool>}  & List of true/false values. \\\hline
		  \camelcase{vByte}  (\nohyphens{inherited}) & \camelcase{byte}  & 8-bit signed integer value. \\\hline
		  \camelcase{vBytes}  & \camelcase{binary}  & List of 8-bit signed integers. \\\hline
		  \camelcase{vDouble}  (\nohyphens{inherited}) & \camelcase{double}  & 64-bit floating point value. \\\hline
		  \camelcase{vDoubles}  & \camelcase{list<double>}  & List of 64-bit floating point values. \\\hline
		  \camelcase{vInteger}  (\nohyphens{inherited}) & \camelcase{i32}  & 32-bit signed integer value. \\\hline
		  \camelcase{vIntegers}  & \camelcase{list<i32>}  & List of 32-bit signed integers. \\\hline
		  \camelcase{vLong}  (\nohyphens{inherited}) & \camelcase{i64}  & 64-bit signed integer value. \\\hline
		  \camelcase{vLongs}  & \camelcase{list<i64>}  & List of 64-bit signed integers. \\\hline
		  \camelcase{vShort}  (\nohyphens{inherited}) & \camelcase{i16}  & 16-bit signed integer value. \\\hline
		  \camelcase{vShorts}  & \camelcase{list<i16>}  & List of 16-bit signed integers. \\\hline
		  \camelcase{vString}  (\nohyphens{inherited}) & \camelcase{string}  & Sequence of UTF8 characters. \\\hline
		  \camelcase{vStrings}  & \camelcase{list<string>}  & List of sequences of UTF8 characters. \\\hline
   \multicolumn{3}{|p{.95\linewidth}|}{\textbf{Used in:} HawkAttributeUpdateEvent, AttributeSlot} \\\hline
\end{longtable}

\subsubsection{Subscription}

Details about a subscription to a topic queue.

\begin{longtable} {|p{4cm}|p{4cm}|p{7.25cm}|}
	\caption{Structure Subscription}
	\label{tab:subscription}\\
	\hline%
	\textbf{Field} & \textbf{Type} & \textbf{Documentation} \\\hline
		  \camelcase{host}  & \camelcase{string}  & Host name of the message queue server. \\\hline
		  \camelcase{port}  & \camelcase{i32}  & Port in which the message queue server is listening. \\\hline
		  \camelcase{queueAddress}  & \camelcase{string}  & Address of the topic queue. \\\hline
		  \camelcase{queueName}  & \camelcase{string}  & Name of the topic queue. \\\hline
   \multicolumn{3}{|p{.95\linewidth}|}{\textbf{Used in:} Hawk.watchStateChanges, Hawk.watchModelChanges} \\\hline
\end{longtable}

\subsubsection{TransformationStatus}

Used to report the status of a long-running transformation by CloudATL.

\begin{longtable} {|p{4cm}|p{4cm}|p{7.25cm}|}
	\caption{Structure TransformationStatus}
	\label{tab:transformationStatus}\\
	\hline%
	\textbf{Field} & \textbf{Type} & \textbf{Documentation} \\\hline
		  \camelcase{elapsed}  & \camelcase{i64}  & Time passed since the start of execution, in milliseconds. \\\hline
		  \camelcase{error}  & \camelcase{string}  & Description of the error that caused the transformation to fail. \\\hline
		  \camelcase{state}  & \camelcase{TransformationState}  & State of the tranformation. \\\hline
   \multicolumn{3}{|p{.95\linewidth}|}{\textbf{Used in:} CloudATL.getStatus} \\\hline
\end{longtable}

\subsubsection{UserProfile}

Minimal details about registered users.

\begin{longtable} {|p{4cm}|p{4cm}|p{7.25cm}|}
	\caption{Structure UserProfile}
	\label{tab:userProfile}\\
	\hline%
	\textbf{Field} & \textbf{Type} & \textbf{Documentation} \\\hline
		  \camelcase{admin}  & \camelcase{bool}  & Whether the user has admin rights (i.e. so that they can create new users, change the status of admin users etc). \\\hline
		  \camelcase{realName}  & \camelcase{string}  & The real name of the user. \\\hline
   \multicolumn{3}{|p{.95\linewidth}|}{\textbf{Used in:} Users.createUser, Users.updateProfile} \\\hline
\end{longtable}

\subsubsection{Value}

Union type for a single scalar value.

\emph{Inherited by}: QueryResult, SlotValue.

\begin{longtable} {|p{4cm}|p{4cm}|p{7.25cm}|}
	\caption{Union Value}
	\label{tab:value}\\
	\hline%
	\textbf{Field} & \textbf{Type} & \textbf{Documentation} \\\hline
		  \camelcase{vBoolean}  & \camelcase{bool}  & Boolean (true/false) value. \\\hline
		  \camelcase{vByte}  & \camelcase{byte}  & 8-bit signed integer value. \\\hline
		  \camelcase{vDouble}  & \camelcase{double}  & 64-bit floating point value. \\\hline
		  \camelcase{vInteger}  & \camelcase{i32}  & 32-bit signed integer value. \\\hline
		  \camelcase{vLong}  & \camelcase{i64}  & 64-bit signed integer value. \\\hline
		  \camelcase{vShort}  & \camelcase{i16}  & 16-bit signed integer value. \\\hline
		  \camelcase{vString}  & \camelcase{string}  & Sequence of UTF8 characters. \\\hline
   
\end{longtable}



\subsection{Enumerations}
\label{sec:enumerations}
\subsubsection{CommitItemChangeType}

\begin{longtable} {|p{4cm}|p{10.5cm}|}
	\caption{Enumeration CommitItemChangeType}
	\label{tab:commitItemChangeType}\\
	\hline%
	\textbf{Name} & \textbf{Documentation} \\\hline 
		ADDED & File was added. \\\hline 
		DELETED & File was removed. \\\hline 
		REPLACED & File was removed. \\\hline 
		UNKNOWN & Unknown type of change. \\\hline 
		UPDATED & File was updated. \\\hline 
	\multicolumn{2}{|p{.95\linewidth}|}{\textbf{Used in:} CommitItem} \\\hline
\end{longtable}

\subsubsection{HawkState}

\begin{longtable} {|p{4cm}|p{10.5cm}|}
	\caption{Enumeration HawkState}
	\label{tab:hawkState}\\
	\hline%
	\textbf{Name} & \textbf{Documentation} \\\hline 
		RUNNING & The instance is running and monitoring the indexed locations. \\\hline 
		STOPPED & The instance is stopped and is not monitoring any indexed locations. \\\hline 
		UPDATING & The instance is updating its contents from the indexed locations. \\\hline 
	\multicolumn{2}{|p{.95\linewidth}|}{\textbf{Used in:} HawkStateEvent, HawkInstance} \\\hline
\end{longtable}

\subsubsection{SubscriptionDurability}

\begin{longtable} {|p{4cm}|p{10.5cm}|}
	\caption{Enumeration SubscriptionDurability}
	\label{tab:subscriptionDurability}\\
	\hline%
	\textbf{Name} & \textbf{Documentation} \\\hline 
		DEFAULT & Subscription survives client disconnections but not server restarts. \\\hline 
		DURABLE & Subscription survives client disconnections and server restarts. \\\hline 
		TEMPORARY & Subscription removed after disconnecting. \\\hline 
	\multicolumn{2}{|p{.95\linewidth}|}{\textbf{Used in:} Hawk.watchModelChanges} \\\hline
\end{longtable}

\subsubsection{TransformationState}

\begin{longtable} {|p{4cm}|p{10.5cm}|}
	\caption{Enumeration TransformationState}
	\label{tab:transformationState}\\
	\hline%
	\textbf{Name} & \textbf{Documentation} \\\hline 
		FAILED & The transformation has failed. \\\hline 
		KILLED & The transformation was interrupted by a user (i.e. killed). \\\hline 
		PREP & The transformation is in preparation. \\\hline 
		RUNNING & The transformation is running. \\\hline 
		SUCCEEDED & The transformation has completed successfully. \\\hline 
	\multicolumn{2}{|p{.95\linewidth}|}{\textbf{Used in:} TransformationStatus} \\\hline
\end{longtable}

\subsection{Exceptions}
\label{sec:exceptions}
\subsubsection{CollaborationRuleNotFound}
No collaboration rule with that name was found for the specified repository.

\begin{longtable} {|p{4cm}|p{4cm}|p{7.25cm}|}
	\caption{Exception CollaborationRuleNotFound}
	\label{tab:collaborationRuleNotFound}\\
	\hline%
	\textbf{Field} & \textbf{Type} & \textbf{Documentation} \\\hline
	    --- & --- & --- \\\hline
   \multicolumn{3}{|p{.95\linewidth}|}{\textbf{Used in:} OfflineCollaboration.removeRule, OfflineCollaboration.updateRule} \\\hline
\end{longtable}
\subsubsection{FailedQuery}
The specified query failed to complete its execution.

\begin{longtable} {|p{4cm}|p{4cm}|p{7.25cm}|}
	\caption{Exception FailedQuery}
	\label{tab:failedQuery}\\
	\hline%
	\textbf{Field} & \textbf{Type} & \textbf{Documentation} \\\hline
		  \camelcase{reason}  & \camelcase{string}  & Reason for the query failing to complete its execution. \\\hline
   \multicolumn{3}{|p{.95\linewidth}|}{\textbf{Used in:} Hawk.query} \\\hline
\end{longtable}
\subsubsection{HawkInstanceNotFound}
No Hawk instance exists with that name.

\begin{longtable} {|p{4cm}|p{4cm}|p{7.25cm}|}
	\caption{Exception HawkInstanceNotFound}
	\label{tab:hawkInstanceNotFound}\\
	\hline%
	\textbf{Field} & \textbf{Type} & \textbf{Documentation} \\\hline
	    --- & --- & --- \\\hline
   \multicolumn{3}{|p{.95\linewidth}|}{\textbf{Used in:} Hawk.removeInstance, Hawk.startInstance, Hawk.stopInstance, Hawk.syncInstance, Hawk.registerMetamodels, Hawk.unregisterMetamodels, Hawk.listMetamodels, Hawk.query, Hawk.resolveProxies, Hawk.addRepository, Hawk.removeRepository, Hawk.updateRepositoryCredentials, Hawk.listRepositories, Hawk.listFiles, Hawk.configurePolling, Hawk.addDerivedAttribute, Hawk.removeDerivedAttribute, Hawk.listDerivedAttributes, Hawk.addIndexedAttribute, Hawk.removeIndexedAttribute, Hawk.listIndexedAttributes, Hawk.getModel, Hawk.watchStateChanges, Hawk.watchModelChanges} \\\hline
\end{longtable}
\subsubsection{HawkInstanceNotRunning}
The selected Hawk instance is not running.

\begin{longtable} {|p{4cm}|p{4cm}|p{7.25cm}|}
	\caption{Exception HawkInstanceNotRunning}
	\label{tab:hawkInstanceNotRunning}\\
	\hline%
	\textbf{Field} & \textbf{Type} & \textbf{Documentation} \\\hline
	    --- & --- & --- \\\hline
   \multicolumn{3}{|p{.95\linewidth}|}{\textbf{Used in:} Hawk.stopInstance, Hawk.syncInstance, Hawk.registerMetamodels, Hawk.unregisterMetamodels, Hawk.listMetamodels, Hawk.query, Hawk.resolveProxies, Hawk.addRepository, Hawk.removeRepository, Hawk.updateRepositoryCredentials, Hawk.listRepositories, Hawk.listFiles, Hawk.configurePolling, Hawk.addDerivedAttribute, Hawk.removeDerivedAttribute, Hawk.listDerivedAttributes, Hawk.addIndexedAttribute, Hawk.removeIndexedAttribute, Hawk.listIndexedAttributes, Hawk.getModel, Hawk.watchStateChanges, Hawk.watchModelChanges} \\\hline
\end{longtable}
\subsubsection{InvalidDerivedAttributeSpec}
The derived attribute specification is not valid.

\begin{longtable} {|p{4cm}|p{4cm}|p{7.25cm}|}
	\caption{Exception InvalidDerivedAttributeSpec}
	\label{tab:invalidDerivedAttributeSpec}\\
	\hline%
	\textbf{Field} & \textbf{Type} & \textbf{Documentation} \\\hline
		  \camelcase{reason}  & \camelcase{string}  & Reason for the spec not being valid. \\\hline
   \multicolumn{3}{|p{.95\linewidth}|}{\textbf{Used in:} Hawk.addDerivedAttribute} \\\hline
\end{longtable}
\subsubsection{InvalidIndexedAttributeSpec}
The indexed attribute specification is not valid.

\begin{longtable} {|p{4cm}|p{4cm}|p{7.25cm}|}
	\caption{Exception InvalidIndexedAttributeSpec}
	\label{tab:invalidIndexedAttributeSpec}\\
	\hline%
	\textbf{Field} & \textbf{Type} & \textbf{Documentation} \\\hline
		  \camelcase{reason}  & \camelcase{string}  & Reason for the spec not being valid. \\\hline
   \multicolumn{3}{|p{.95\linewidth}|}{\textbf{Used in:} Hawk.addIndexedAttribute} \\\hline
\end{longtable}
\subsubsection{InvalidMetamodel}
The provided metamodel is not valid (e.g. unparsable or inconsistent).

\begin{longtable} {|p{4cm}|p{4cm}|p{7.25cm}|}
	\caption{Exception InvalidMetamodel}
	\label{tab:invalidMetamodel}\\
	\hline%
	\textbf{Field} & \textbf{Type} & \textbf{Documentation} \\\hline
		  \camelcase{reason}  & \camelcase{string}  & Reason for the metamodel not being valid. \\\hline
   \multicolumn{3}{|p{.95\linewidth}|}{\textbf{Used in:} Hawk.registerMetamodels} \\\hline
\end{longtable}
\subsubsection{InvalidModelSpec}
The model specification is not valid: the model or the metamodels are inaccessible or invalid.

\begin{longtable} {|p{4cm}|p{4cm}|p{7.25cm}|}
	\caption{Exception InvalidModelSpec}
	\label{tab:invalidModelSpec}\\
	\hline%
	\textbf{Field} & \textbf{Type} & \textbf{Documentation} \\\hline
		  \camelcase{reason}  & \camelcase{string}  & Reason for the spec not being valid. \\\hline
		  \camelcase{spec}  & \camelcase{ModelSpec}  & A copy of the invalid model specification. \\\hline
   \multicolumn{3}{|p{.95\linewidth}|}{\textbf{Used in:} CloudATL.launch} \\\hline
\end{longtable}
\subsubsection{InvalidPollingConfiguration}
The polling configuration is not valid.

\begin{longtable} {|p{4cm}|p{4cm}|p{7.25cm}|}
	\caption{Exception InvalidPollingConfiguration}
	\label{tab:invalidPollingConfiguration}\\
	\hline%
	\textbf{Field} & \textbf{Type} & \textbf{Documentation} \\\hline
		  \camelcase{reason}  & \camelcase{string}  & Reason for the spec not being valid. \\\hline
   \multicolumn{3}{|p{.95\linewidth}|}{\textbf{Used in:} Hawk.configurePolling} \\\hline
\end{longtable}
\subsubsection{InvalidQuery}
The specified query is not valid.

\begin{longtable} {|p{4cm}|p{4cm}|p{7.25cm}|}
	\caption{Exception InvalidQuery}
	\label{tab:invalidQuery}\\
	\hline%
	\textbf{Field} & \textbf{Type} & \textbf{Documentation} \\\hline
		  \camelcase{reason}  & \camelcase{string}  & Reason for the query not being valid. \\\hline
   \multicolumn{3}{|p{.95\linewidth}|}{\textbf{Used in:} Hawk.query} \\\hline
\end{longtable}
\subsubsection{InvalidTransformation}
The transformation is not valid: it is unparsable or inconsistent.

\begin{longtable} {|p{4cm}|p{4cm}|p{7.25cm}|}
	\caption{Exception InvalidTransformation}
	\label{tab:invalidTransformation}\\
	\hline%
	\textbf{Field} & \textbf{Type} & \textbf{Documentation} \\\hline
		  \camelcase{location}  & \camelcase{string}  & Location of the problem, if applicable. Usually a combination of line and column numbers. \\\hline
		  \camelcase{reason}  & \camelcase{string}  & Reason for the transformation not being valid. \\\hline
   \multicolumn{3}{|p{.95\linewidth}|}{\textbf{Used in:} CloudATL.launch} \\\hline
\end{longtable}
\subsubsection{TransformationTokenNotFound}
The specified transformation token does not exist within the invokved MONDO instance.

\begin{longtable} {|p{4cm}|p{4cm}|p{7.25cm}|}
	\caption{Exception TransformationTokenNotFound}
	\label{tab:transformationTokenNotFound}\\
	\hline%
	\textbf{Field} & \textbf{Type} & \textbf{Documentation} \\\hline
		  \camelcase{token}  & \camelcase{string}  & Transformation token which was not found within the invoked MONDO instance. \\\hline
   \multicolumn{3}{|p{.95\linewidth}|}{\textbf{Used in:} CloudATL.getStatus, CloudATL.kill} \\\hline
\end{longtable}
\subsubsection{UnknownQueryLanguage}
The specified query language is not supported by the operation.

\begin{longtable} {|p{4cm}|p{4cm}|p{7.25cm}|}
	\caption{Exception UnknownQueryLanguage}
	\label{tab:unknownQueryLanguage}\\
	\hline%
	\textbf{Field} & \textbf{Type} & \textbf{Documentation} \\\hline
	    --- & --- & --- \\\hline
   \multicolumn{3}{|p{.95\linewidth}|}{\textbf{Used in:} Hawk.query} \\\hline
\end{longtable}
\subsubsection{UnknownRepositoryType}
The specified repository type is not supported by the operation.

\begin{longtable} {|p{4cm}|p{4cm}|p{7.25cm}|}
	\caption{Exception UnknownRepositoryType}
	\label{tab:unknownRepositoryType}\\
	\hline%
	\textbf{Field} & \textbf{Type} & \textbf{Documentation} \\\hline
	    --- & --- & --- \\\hline
   \multicolumn{3}{|p{.95\linewidth}|}{\textbf{Used in:} Hawk.addRepository} \\\hline
\end{longtable}
\subsubsection{UserExists}
The specified username already exists.

\begin{longtable} {|p{4cm}|p{4cm}|p{7.25cm}|}
	\caption{Exception UserExists}
	\label{tab:userExists}\\
	\hline%
	\textbf{Field} & \textbf{Type} & \textbf{Documentation} \\\hline
	    --- & --- & --- \\\hline
   \multicolumn{3}{|p{.95\linewidth}|}{\textbf{Used in:} Users.createUser} \\\hline
\end{longtable}
\subsubsection{UserNotFound}
The specified username does not exist.

\begin{longtable} {|p{4cm}|p{4cm}|p{7.25cm}|}
	\caption{Exception UserNotFound}
	\label{tab:userNotFound}\\
	\hline%
	\textbf{Field} & \textbf{Type} & \textbf{Documentation} \\\hline
	    --- & --- & --- \\\hline
   \multicolumn{3}{|p{.95\linewidth}|}{\textbf{Used in:} Users.updateProfile, Users.updatePassword, Users.deleteUser} \\\hline
\end{longtable}
\subsubsection{VCSAuthenticationFailed}
The client failed to prove its identity in the VCS.

\begin{longtable} {|p{4cm}|p{4cm}|p{7.25cm}|}
	\caption{Exception VCSAuthenticationFailed}
	\label{tab:vCSAuthenticationFailed}\\
	\hline%
	\textbf{Field} & \textbf{Type} & \textbf{Documentation} \\\hline
	    --- & --- & --- \\\hline
   \multicolumn{3}{|p{.95\linewidth}|}{\textbf{Used in:} Hawk.addRepository} \\\hline
\end{longtable}
