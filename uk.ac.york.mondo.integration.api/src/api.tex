% WARNING: This is an auto-generated file. Do not modify manually.
\makeatletter
  \newcommand{\camelhyph}[1]{\@fterfirst\c@amelhyph#1\relax }
  \def\@fterfirst #1#2{#2#1}
  \def\c@amelhyph #1{%
   \ifthenelse{\equal{#1}\relax}{}{%  Do nothing if the end has been reached
     \ifnum`#1<91 \-#1\else#1\fi%     Check whether #1 is an uppercase letter,
                                %     if so, print \-#1, otherwise #1
    \expandafter\c@amelhyph%    %     insert \c@amelhyph again.
}}
\makeatother

\newcommand{\camelcase}[1]{\camelhyph{#1}\xspace}

\subsection{Services}
\label{sec:services}


The present section lists the services offered by the current version of the
MONDO platform. For each service, the available operations are presented, along
with the parameters they take, their return type and the exceptions they may
throw if they fail.

In addition to their explicitly listed exceptions, operations requiring authentication
may throw an AuthenticationFailed exception if the user fails to provide valid credentials
or does not have the required permissions. In its current version, the MONDO platform
assumes that authenticated users will have the necessary permissions. The deliverable D6.3
in M24 will include a more fine-grained permissions system which separates authentication and
authorization.

\subsubsection{Users}
\label{sec:Users}
The majority of service operations provided by the MONDO
		platform require user authentication (indicated in the top-left
		cell of each operation table) to prevent unaccountable use.
		As such, the platform needs to provide basic user management service operations
		for creating, updating and deleting user accounts.

\begin{longtable} {|p{2cm}|p{3.5cm}|p{2.5cm}|p{6.5cm}|}
	\caption{Operation Users.createUser}
	\label{tab:users-createUser}\\
	\hline%
	Auth: Yes & \textbf{Name} & \textbf{Type} & \textbf{Documentation} \\\hline 
	\textbf{Operation} & \camelcase{createUser} & void  & Creates a new platform user. \\\hline
	\textbf{Parameter}s & & & \\\hline
	 & username & \camelcase{String} & A unique identifier for the user. \\\hline 
	 & password & \camelcase{String} & The desired password. \\\hline 
	 & profile & \camelcase{UserProfile} & The profile of the user. \\\hline 
	\textbf{Exception}s & & & \\\hline
	 & \camelcase{UserExists} & \multicolumn{2}{p{8.5cm}|}{The specified username already exists.} \\\hline 
\end{longtable}
\begin{longtable} {|p{2cm}|p{3.5cm}|p{2.5cm}|p{6.5cm}|}
	\caption{Operation Users.updateProfile}
	\label{tab:users-updateProfile}\\
	\hline%
	Auth: Yes & \textbf{Name} & \textbf{Type} & \textbf{Documentation} \\\hline 
	\textbf{Operation} & \camelcase{updateProfile} & void  & Updates the profile of a platform user. \\\hline
	\textbf{Parameter}s & & & \\\hline
	 & username & \camelcase{String} & The name of the user to update the profile of. \\\hline 
	 & profile & \camelcase{UserProfile} & The updated profile of the user. \\\hline 
	\textbf{Exception}s & & & \\\hline
	 & \camelcase{UserNotFound} & \multicolumn{2}{p{8.5cm}|}{The specified username does not exist.} \\\hline 
\end{longtable}
\begin{longtable} {|p{2cm}|p{3.5cm}|p{2.5cm}|p{6.5cm}|}
	\caption{Operation Users.updatePassword}
	\label{tab:users-updatePassword}\\
	\hline%
	Auth: Yes & \textbf{Name} & \textbf{Type} & \textbf{Documentation} \\\hline 
	\textbf{Operation} & \camelcase{updatePassword} & void  & Updates the password of a platform user. \\\hline
	\textbf{Parameter}s & & & \\\hline
	 & username & \camelcase{String} & The name of the user to update the profile of. \\\hline 
	 & newPassword & \camelcase{String} & New password to be set. \\\hline 
	\textbf{Exception}s & & & \\\hline
	 & \camelcase{UserNotFound} & \multicolumn{2}{p{8.5cm}|}{The specified username does not exist.} \\\hline 
\end{longtable}
\begin{longtable} {|p{2cm}|p{3.5cm}|p{2.5cm}|p{6.5cm}|}
	\caption{Operation Users.deleteUser}
	\label{tab:users-deleteUser}\\
	\hline%
	Auth: Yes & \textbf{Name} & \textbf{Type} & \textbf{Documentation} \\\hline 
	\textbf{Operation} & \camelcase{deleteUser} & void  & Deletes a platform user. \\\hline
	\textbf{Parameter}s & & & \\\hline
	 & username & \camelcase{String} & The name of the user to delete. \\\hline 
	\textbf{Exception}s & & & \\\hline
	 & \camelcase{UserNotFound} & \multicolumn{2}{p{8.5cm}|}{The specified username does not exist.} \\\hline 
\end{longtable}

\subsubsection{Hawk}
\label{sec:Hawk}
The following service operations expose the capabilities of the Hawk heterogeneous model indexing
framework developed in Work Package 5. The framework is discussed in detail in D5.2 and D5.3.

\begin{longtable} {|p{2cm}|p{3.5cm}|p{2.5cm}|p{6.5cm}|}
	\caption{Operation Hawk.createInstance}
	\label{tab:hawk-createInstance}\\
	\hline%
	Auth: Yes & \textbf{Name} & \textbf{Type} & \textbf{Documentation} \\\hline 
	\textbf{Operation} & \camelcase{createInstance} & void  & Creates a new Hawk instance (stopped). \\\hline
	\textbf{Parameter}s & & & \\\hline
	 & name & \camelcase{String} & The unique name of the new Hawk instance. \\\hline 
	 & backend & \camelcase{String} & The name of the backend to be used. \\\hline 
	 & minimumDelayMillis & \camelcase{int} & Minimum delay between periodic synchronization in milliseconds. \\\hline 
	 & maximumDelayMillis & \camelcase{int} & Maximum delay between periodic synchronization in milliseconds (0 to disable periodic synchronization). \\\hline 
\end{longtable}
\begin{longtable} {|p{2cm}|p{3.5cm}|p{2.5cm}|p{6.5cm}|}
	\caption{Operation Hawk.listBackends}
	\label{tab:hawk-listBackends}\\
	\hline%
	Auth: Yes & \textbf{Name} & \textbf{Type} & \textbf{Documentation} \\\hline 
	\textbf{Operation} & \camelcase{listBackends} & \camelcase{String[]}  & Lists the names of the available storage backends. \\\hline
	\textbf{Parameter}s & & & \\\hline
\end{longtable}
\begin{longtable} {|p{2cm}|p{3.5cm}|p{2.5cm}|p{6.5cm}|}
	\caption{Operation Hawk.listInstances}
	\label{tab:hawk-listInstances}\\
	\hline%
	Auth: Yes & \textbf{Name} & \textbf{Type} & \textbf{Documentation} \\\hline 
	\textbf{Operation} & \camelcase{listInstances} & \camelcase{HawkInstance[]}  & Lists the details of all Hawk instances. \\\hline
	\textbf{Parameter}s & & & \\\hline
\end{longtable}
\begin{longtable} {|p{2cm}|p{3.5cm}|p{2.5cm}|p{6.5cm}|}
	\caption{Operation Hawk.removeInstance}
	\label{tab:hawk-removeInstance}\\
	\hline%
	Auth: Yes & \textbf{Name} & \textbf{Type} & \textbf{Documentation} \\\hline 
	\textbf{Operation} & \camelcase{removeInstance} & void  & Removes an existing Hawk instance. \\\hline
	\textbf{Parameter}s & & & \\\hline
	 & name & \camelcase{String} & The name of the Hawk instance to remove. \\\hline 
	\textbf{Exception}s & & & \\\hline
	 & \camelcase{HawkInstanceNotFound} & \multicolumn{2}{p{8.5cm}|}{No Hawk instance exists with that name.} \\\hline 
\end{longtable}
\begin{longtable} {|p{2cm}|p{3.5cm}|p{2.5cm}|p{6.5cm}|}
	\caption{Operation Hawk.startInstance}
	\label{tab:hawk-startInstance}\\
	\hline%
	Auth: Yes & \textbf{Name} & \textbf{Type} & \textbf{Documentation} \\\hline 
	\textbf{Operation} & \camelcase{startInstance} & void  & Starts a stopped Hawk instance. \\\hline
	\textbf{Parameter}s & & & \\\hline
	 & name & \camelcase{String} & The name of the Hawk instance to start. \\\hline 
	\textbf{Exception}s & & & \\\hline
	 & \camelcase{HawkInstanceNotFound} & \multicolumn{2}{p{8.5cm}|}{No Hawk instance exists with that name.} \\\hline 
\end{longtable}
\begin{longtable} {|p{2cm}|p{3.5cm}|p{2.5cm}|p{6.5cm}|}
	\caption{Operation Hawk.stopInstance}
	\label{tab:hawk-stopInstance}\\
	\hline%
	Auth: Yes & \textbf{Name} & \textbf{Type} & \textbf{Documentation} \\\hline 
	\textbf{Operation} & \camelcase{stopInstance} & void  & Stops a running Hawk instance. \\\hline
	\textbf{Parameter}s & & & \\\hline
	 & name & \camelcase{String} & The name of the Hawk instance to stop. \\\hline 
	\textbf{Exception}s & & & \\\hline
	 & \camelcase{HawkInstanceNotFound} & \multicolumn{2}{p{8.5cm}|}{No Hawk instance exists with that name.} \\\hline 
	 & \camelcase{HawkInstanceNotRunning} & \multicolumn{2}{p{8.5cm}|}{The selected Hawk instance is not running.} \\\hline 
\end{longtable}
\begin{longtable} {|p{2cm}|p{3.5cm}|p{2.5cm}|p{6.5cm}|}
	\caption{Operation Hawk.syncInstance}
	\label{tab:hawk-syncInstance}\\
	\hline%
	Auth: Yes & \textbf{Name} & \textbf{Type} & \textbf{Documentation} \\\hline 
	\textbf{Operation} & \camelcase{syncInstance} & void  & Forces an immediate synchronization on a Hawk instance. \\\hline
	\textbf{Parameter}s & & & \\\hline
	 & name & \camelcase{String} & The name of the Hawk instance to stop. \\\hline 
	\textbf{Exception}s & & & \\\hline
	 & \camelcase{HawkInstanceNotFound} & \multicolumn{2}{p{8.5cm}|}{No Hawk instance exists with that name.} \\\hline 
	 & \camelcase{HawkInstanceNotRunning} & \multicolumn{2}{p{8.5cm}|}{The selected Hawk instance is not running.} \\\hline 
\end{longtable}
\begin{longtable} {|p{2cm}|p{3.5cm}|p{2.5cm}|p{6.5cm}|}
	\caption{Operation Hawk.registerMetamodels}
	\label{tab:hawk-registerMetamodels}\\
	\hline%
	Auth: Yes & \textbf{Name} & \textbf{Type} & \textbf{Documentation} \\\hline 
	\textbf{Operation} & \camelcase{registerMetamodels} & void  & Registers a set of file-based metamodels with a Hawk instance. \\\hline
	\textbf{Parameter}s & & & \\\hline
	 & name & \camelcase{String} & The name of the Hawk instance. \\\hline 
	 & metamodel & \camelcase{File[]} & The metamodels to register.
	                                    			More than one metamodel files can be provided in one
	                                    			go to accomodate fragmented metamodels. \\\hline 
	\textbf{Exception}s & & & \\\hline
	 & \camelcase{HawkInstanceNotFound} & \multicolumn{2}{p{8.5cm}|}{No Hawk instance exists with that name.} \\\hline 
	 & \camelcase{InvalidMetamodel} & \multicolumn{2}{p{8.5cm}|}{The provided metamodel is not valid (e.g. unparsable or inconsistent).} \\\hline 
	 & \camelcase{HawkInstanceNotRunning} & \multicolumn{2}{p{8.5cm}|}{The selected Hawk instance is not running.} \\\hline 
\end{longtable}
\begin{longtable} {|p{2cm}|p{3.5cm}|p{2.5cm}|p{6.5cm}|}
	\caption{Operation Hawk.unregisterMetamodels}
	\label{tab:hawk-unregisterMetamodels}\\
	\hline%
	Auth: Yes & \textbf{Name} & \textbf{Type} & \textbf{Documentation} \\\hline 
	\textbf{Operation} & \camelcase{unregisterMetamodels} & void  & Unregisters a metamodel from a Hawk instance. \\\hline
	\textbf{Parameter}s & & & \\\hline
	 & name & \camelcase{String} & The name of the Hawk instance. \\\hline 
	 & metamodel & \camelcase{String[]} & The URIs of the metamodels. \\\hline 
	\textbf{Exception}s & & & \\\hline
	 & \camelcase{HawkInstanceNotFound} & \multicolumn{2}{p{8.5cm}|}{No Hawk instance exists with that name.} \\\hline 
	 & \camelcase{HawkInstanceNotRunning} & \multicolumn{2}{p{8.5cm}|}{The selected Hawk instance is not running.} \\\hline 
\end{longtable}
\begin{longtable} {|p{2cm}|p{3.5cm}|p{2.5cm}|p{6.5cm}|}
	\caption{Operation Hawk.listMetamodels}
	\label{tab:hawk-listMetamodels}\\
	\hline%
	Auth: Yes & \textbf{Name} & \textbf{Type} & \textbf{Documentation} \\\hline 
	\textbf{Operation} & \camelcase{listMetamodels} & \camelcase{String[]}  & Lists the URIs of the registered metamodels of a Hawk instance. \\\hline
	\textbf{Parameter}s & & & \\\hline
	 & name & \camelcase{String} & The name of the Hawk instance. \\\hline 
	\textbf{Exception}s & & & \\\hline
	 & \camelcase{HawkInstanceNotFound} & \multicolumn{2}{p{8.5cm}|}{No Hawk instance exists with that name.} \\\hline 
	 & \camelcase{HawkInstanceNotRunning} & \multicolumn{2}{p{8.5cm}|}{The selected Hawk instance is not running.} \\\hline 
\end{longtable}
\begin{longtable} {|p{2cm}|p{3.5cm}|p{2.5cm}|p{6.5cm}|}
	\caption{Operation Hawk.listQueryLanguages}
	\label{tab:hawk-listQueryLanguages}\\
	\hline%
	Auth: Yes & \textbf{Name} & \textbf{Type} & \textbf{Documentation} \\\hline 
	\textbf{Operation} & \camelcase{listQueryLanguages} & \camelcase{String[]}  & Lists the supported query languages and their status. \\\hline
	\textbf{Parameter}s & & & \\\hline
	 & name & \camelcase{String} & The name of the Hawk instance. \\\hline 
\end{longtable}
\begin{longtable} {|p{2cm}|p{3.5cm}|p{2.5cm}|p{6.5cm}|}
	\caption{Operation Hawk.query}
	\label{tab:hawk-query}\\
	\hline%
	Auth: Yes & \textbf{Name} & \textbf{Type} & \textbf{Documentation} \\\hline 
	\textbf{Operation} & \camelcase{query} & \camelcase{QueryResult[]}  & Runs a query on a Hawk instance and returns a sequence of scalar values and/or model elements. \\\hline
	\textbf{Parameter}s & & & \\\hline
	 & name & \camelcase{String} & The name of the Hawk instance. \\\hline 
	 & query & \camelcase{String} & The query to be executed. \\\hline 
	 & language & \camelcase{String} & The name of the query language used (e.g. EOL, OCL). \\\hline 
	 & repositoryPattern & \camelcase{String} & The repository for the query (or * for all repositories). \\\hline 
	 & filePatterns & \camelcase{String[]} & The file patterns for the query (e.g. *.uml). \\\hline 
	 & includeAttributes  (optional)& \camelcase{EBooleanObject} & Whether to include attributes (true) or not (false) in model element results. \\\hline 
	 & includeReferences  (optional)& \camelcase{EBooleanObject} & Whether to include references (true) or not (false) in model element results. \\\hline 
	 & includeNodeIDs  (optional)& \camelcase{EBooleanObject} & Whether to include node IDs (true) or not (false) in model element results. \\\hline 
	 & includeContained  (optional)& \camelcase{EBooleanObject} & Whether to include all the child elements of the model element results (true) or not (false). \\\hline 
	\textbf{Exception}s & & & \\\hline
	 & \camelcase{HawkInstanceNotFound} & \multicolumn{2}{p{8.5cm}|}{No Hawk instance exists with that name.} \\\hline 
	 & \camelcase{HawkInstanceNotRunning} & \multicolumn{2}{p{8.5cm}|}{The selected Hawk instance is not running.} \\\hline 
	 & \camelcase{UnknownQueryLanguage} & \multicolumn{2}{p{8.5cm}|}{The specified query language is not supported by the operation.} \\\hline 
	 & \camelcase{InvalidQuery} & \multicolumn{2}{p{8.5cm}|}{The specified query is not valid.} \\\hline 
	 & \camelcase{FailedQuery} & \multicolumn{2}{p{8.5cm}|}{The specified query failed to complete its execution.} \\\hline 
\end{longtable}
\begin{longtable} {|p{2cm}|p{3.5cm}|p{2.5cm}|p{6.5cm}|}
	\caption{Operation Hawk.resolveProxies}
	\label{tab:hawk-resolveProxies}\\
	\hline%
	Auth: Yes & \textbf{Name} & \textbf{Type} & \textbf{Documentation} \\\hline 
	\textbf{Operation} & \camelcase{resolveProxies} & \camelcase{ModelElement[]}  & Returns populated model elements for the provided proxies. \\\hline
	\textbf{Parameter}s & & & \\\hline
	 & name & \camelcase{String} & The name of the Hawk instance. \\\hline 
	 & ids & \camelcase{String[]} & Proxy model element IDs to be resolved. \\\hline 
	 & includeAttributes  (optional)& \camelcase{EBooleanObject} & Whether to include attributes (true) or not (false). \\\hline 
	 & includeReferences  (optional)& \camelcase{EBooleanObject} & Whether to include references (true) or not (false). \\\hline 
	\textbf{Exception}s & & & \\\hline
	 & \camelcase{HawkInstanceNotFound} & \multicolumn{2}{p{8.5cm}|}{No Hawk instance exists with that name.} \\\hline 
	 & \camelcase{HawkInstanceNotRunning} & \multicolumn{2}{p{8.5cm}|}{The selected Hawk instance is not running.} \\\hline 
\end{longtable}
\begin{longtable} {|p{2cm}|p{3.5cm}|p{2.5cm}|p{6.5cm}|}
	\caption{Operation Hawk.addRepository}
	\label{tab:hawk-addRepository}\\
	\hline%
	Auth: Yes & \textbf{Name} & \textbf{Type} & \textbf{Documentation} \\\hline 
	\textbf{Operation} & \camelcase{addRepository} & void  & Asks a Hawk instance to start monitoring a repository. \\\hline
	\textbf{Parameter}s & & & \\\hline
	 & name & \camelcase{String} & The name of the Hawk instance. \\\hline 
	 & repo & \camelcase{Repository} & The repository to monitor. \\\hline 
	 & credentials  (optional)& \camelcase{Credentials} & A valid set of credentials that has read-access to the repository. \\\hline 
	\textbf{Exception}s & & & \\\hline
	 & \camelcase{HawkInstanceNotFound} & \multicolumn{2}{p{8.5cm}|}{No Hawk instance exists with that name.} \\\hline 
	 & \camelcase{HawkInstanceNotRunning} & \multicolumn{2}{p{8.5cm}|}{The selected Hawk instance is not running.} \\\hline 
	 & \camelcase{UnknownRepositoryType} & \multicolumn{2}{p{8.5cm}|}{The specified repository type is not supported by the operation.} \\\hline 
	 & \camelcase{VCSAuthenticationFailed} & \multicolumn{2}{p{8.5cm}|}{The client failed to prove its identity in the VCS.} \\\hline 
\end{longtable}
\begin{longtable} {|p{2cm}|p{3.5cm}|p{2.5cm}|p{6.5cm}|}
	\caption{Operation Hawk.removeRepository}
	\label{tab:hawk-removeRepository}\\
	\hline%
	Auth: Yes & \textbf{Name} & \textbf{Type} & \textbf{Documentation} \\\hline 
	\textbf{Operation} & \camelcase{removeRepository} & void  & Asks a Hawk instance to stop monitoring a repository. \\\hline
	\textbf{Parameter}s & & & \\\hline
	 & name & \camelcase{String} & The name of the Hawk instance. \\\hline 
	 & uri & \camelcase{String} & The URI of the repository to stop monitoring. \\\hline 
	\textbf{Exception}s & & & \\\hline
	 & \camelcase{HawkInstanceNotFound} & \multicolumn{2}{p{8.5cm}|}{No Hawk instance exists with that name.} \\\hline 
	 & \camelcase{HawkInstanceNotRunning} & \multicolumn{2}{p{8.5cm}|}{The selected Hawk instance is not running.} \\\hline 
\end{longtable}
\begin{longtable} {|p{2cm}|p{3.5cm}|p{2.5cm}|p{6.5cm}|}
	\caption{Operation Hawk.updateRepositoryCredentials}
	\label{tab:hawk-updateRepositoryCredentials}\\
	\hline%
	Auth: Yes & \textbf{Name} & \textbf{Type} & \textbf{Documentation} \\\hline 
	\textbf{Operation} & \camelcase{updateRepositoryCredentials} & void  & Changes the credentials used to monitor a repository. \\\hline
	\textbf{Parameter}s & & & \\\hline
	 & name & \camelcase{String} & The name of the Hawk instance. \\\hline 
	 & uri & \camelcase{String} & The URI of the repository to update. \\\hline 
	 & cred & \camelcase{Credentials} & The new credentials to be used. \\\hline 
	\textbf{Exception}s & & & \\\hline
	 & \camelcase{HawkInstanceNotFound} & \multicolumn{2}{p{8.5cm}|}{No Hawk instance exists with that name.} \\\hline 
	 & \camelcase{HawkInstanceNotRunning} & \multicolumn{2}{p{8.5cm}|}{The selected Hawk instance is not running.} \\\hline 
\end{longtable}
\begin{longtable} {|p{2cm}|p{3.5cm}|p{2.5cm}|p{6.5cm}|}
	\caption{Operation Hawk.listRepositories}
	\label{tab:hawk-listRepositories}\\
	\hline%
	Auth: Yes & \textbf{Name} & \textbf{Type} & \textbf{Documentation} \\\hline 
	\textbf{Operation} & \camelcase{listRepositories} & \camelcase{Repository[]}  & Lists the repositories monitored by a Hawk instance. \\\hline
	\textbf{Parameter}s & & & \\\hline
	 & name & \camelcase{String} & The name of the Hawk instance. \\\hline 
	\textbf{Exception}s & & & \\\hline
	 & \camelcase{HawkInstanceNotFound} & \multicolumn{2}{p{8.5cm}|}{No Hawk instance exists with that name.} \\\hline 
	 & \camelcase{HawkInstanceNotRunning} & \multicolumn{2}{p{8.5cm}|}{The selected Hawk instance is not running.} \\\hline 
\end{longtable}
\begin{longtable} {|p{2cm}|p{3.5cm}|p{2.5cm}|p{6.5cm}|}
	\caption{Operation Hawk.listRepositoryTypes}
	\label{tab:hawk-listRepositoryTypes}\\
	\hline%
	Auth: Yes & \textbf{Name} & \textbf{Type} & \textbf{Documentation} \\\hline 
	\textbf{Operation} & \camelcase{listRepositoryTypes} & \camelcase{String[]}  & Lists the available repository types in this installation. \\\hline
	\textbf{Parameter}s & & & \\\hline
\end{longtable}
\begin{longtable} {|p{2cm}|p{3.5cm}|p{2.5cm}|p{6.5cm}|}
	\caption{Operation Hawk.listFiles}
	\label{tab:hawk-listFiles}\\
	\hline%
	Auth: Yes & \textbf{Name} & \textbf{Type} & \textbf{Documentation} \\\hline 
	\textbf{Operation} & \camelcase{listFiles} & \camelcase{String[]}  & Lists the paths of the files of the indexed repository. \\\hline
	\textbf{Parameter}s & & & \\\hline
	 & name & \camelcase{String} & The name of the Hawk instance. \\\hline 
	 & repository & \camelcase{String[]} & The URI of the indexed repository. \\\hline 
	 & filePatterns & \camelcase{String[]} & File name patterns to search for (* lists all files). \\\hline 
	\textbf{Exception}s & & & \\\hline
	 & \camelcase{HawkInstanceNotFound} & \multicolumn{2}{p{8.5cm}|}{No Hawk instance exists with that name.} \\\hline 
	 & \camelcase{HawkInstanceNotRunning} & \multicolumn{2}{p{8.5cm}|}{The selected Hawk instance is not running.} \\\hline 
\end{longtable}
\begin{longtable} {|p{2cm}|p{3.5cm}|p{2.5cm}|p{6.5cm}|}
	\caption{Operation Hawk.configurePolling}
	\label{tab:hawk-configurePolling}\\
	\hline%
	Auth: Yes & \textbf{Name} & \textbf{Type} & \textbf{Documentation} \\\hline 
	\textbf{Operation} & \camelcase{configurePolling} & void  & Sets the base polling period and max interval of a Hawk instance. \\\hline
	\textbf{Parameter}s & & & \\\hline
	 & name & \camelcase{String} & The name of the Hawk instance. \\\hline 
	 & base & \camelcase{int} & The base polling period (in seconds). \\\hline 
	 & max & \camelcase{int} & The maximum polling interval (in seconds). \\\hline 
	\textbf{Exception}s & & & \\\hline
	 & \camelcase{HawkInstanceNotFound} & \multicolumn{2}{p{8.5cm}|}{No Hawk instance exists with that name.} \\\hline 
	 & \camelcase{HawkInstanceNotRunning} & \multicolumn{2}{p{8.5cm}|}{The selected Hawk instance is not running.} \\\hline 
	 & \camelcase{InvalidPollingConfiguration} & \multicolumn{2}{p{8.5cm}|}{The polling configuration is not valid.} \\\hline 
\end{longtable}
\begin{longtable} {|p{2cm}|p{3.5cm}|p{2.5cm}|p{6.5cm}|}
	\caption{Operation Hawk.addDerivedAttribute}
	\label{tab:hawk-addDerivedAttribute}\\
	\hline%
	Auth: Yes & \textbf{Name} & \textbf{Type} & \textbf{Documentation} \\\hline 
	\textbf{Operation} & \camelcase{addDerivedAttribute} & void  & Add a new derived attribute to a Hawk instance. \\\hline
	\textbf{Parameter}s & & & \\\hline
	 & name & \camelcase{String} & The name of the Hawk instance. \\\hline 
	 & spec & \camelcase{DerivedAttributeSpec} & The details of the new derived attribute. \\\hline 
	\textbf{Exception}s & & & \\\hline
	 & \camelcase{HawkInstanceNotFound} & \multicolumn{2}{p{8.5cm}|}{No Hawk instance exists with that name.} \\\hline 
	 & \camelcase{HawkInstanceNotRunning} & \multicolumn{2}{p{8.5cm}|}{The selected Hawk instance is not running.} \\\hline 
	 & \camelcase{InvalidDerivedAttributeSpec} & \multicolumn{2}{p{8.5cm}|}{The derived attribute specification is not valid.} \\\hline 
\end{longtable}
\begin{longtable} {|p{2cm}|p{3.5cm}|p{2.5cm}|p{6.5cm}|}
	\caption{Operation Hawk.removeDerivedAttribute}
	\label{tab:hawk-removeDerivedAttribute}\\
	\hline%
	Auth: Yes & \textbf{Name} & \textbf{Type} & \textbf{Documentation} \\\hline 
	\textbf{Operation} & \camelcase{removeDerivedAttribute} & void  & Remove a derived attribute from a Hawk instance. \\\hline
	\textbf{Parameter}s & & & \\\hline
	 & name & \camelcase{String} & The name of the Hawk instance. \\\hline 
	 & spec & \camelcase{DerivedAttributeSpec} & The details of the derived attribute to be removed.
	                                             			Only the first three fields of the spec
	                                             			need to be populated. \\\hline 
	\textbf{Exception}s & & & \\\hline
	 & \camelcase{HawkInstanceNotFound} & \multicolumn{2}{p{8.5cm}|}{No Hawk instance exists with that name.} \\\hline 
	 & \camelcase{HawkInstanceNotRunning} & \multicolumn{2}{p{8.5cm}|}{The selected Hawk instance is not running.} \\\hline 
\end{longtable}
\begin{longtable} {|p{2cm}|p{3.5cm}|p{2.5cm}|p{6.5cm}|}
	\caption{Operation Hawk.listDerivedAttributes}
	\label{tab:hawk-listDerivedAttributes}\\
	\hline%
	Auth: Yes & \textbf{Name} & \textbf{Type} & \textbf{Documentation} \\\hline 
	\textbf{Operation} & \camelcase{listDerivedAttributes} & \camelcase{DerivedAttributeSpec[]}  & Lists the derived attributes of a Hawk instance. Only the first three fields of the spec are currently populated. \\\hline
	\textbf{Parameter}s & & & \\\hline
	 & name & \camelcase{String} & The name of the Hawk instance. \\\hline 
	\textbf{Exception}s & & & \\\hline
	 & \camelcase{HawkInstanceNotFound} & \multicolumn{2}{p{8.5cm}|}{No Hawk instance exists with that name.} \\\hline 
	 & \camelcase{HawkInstanceNotRunning} & \multicolumn{2}{p{8.5cm}|}{The selected Hawk instance is not running.} \\\hline 
\end{longtable}
\begin{longtable} {|p{2cm}|p{3.5cm}|p{2.5cm}|p{6.5cm}|}
	\caption{Operation Hawk.addIndexedAttribute}
	\label{tab:hawk-addIndexedAttribute}\\
	\hline%
	Auth: Yes & \textbf{Name} & \textbf{Type} & \textbf{Documentation} \\\hline 
	\textbf{Operation} & \camelcase{addIndexedAttribute} & void  & Add a new indexed attribute to a Hawk instance. \\\hline
	\textbf{Parameter}s & & & \\\hline
	 & name & \camelcase{String} & The name of the Hawk instance. \\\hline 
	 & spec & \camelcase{IndexedAttributeSpec} & The details of the new indexed attribute. \\\hline 
	\textbf{Exception}s & & & \\\hline
	 & \camelcase{HawkInstanceNotFound} & \multicolumn{2}{p{8.5cm}|}{No Hawk instance exists with that name.} \\\hline 
	 & \camelcase{HawkInstanceNotRunning} & \multicolumn{2}{p{8.5cm}|}{The selected Hawk instance is not running.} \\\hline 
	 & \camelcase{InvalidIndexedAttributeSpec} & \multicolumn{2}{p{8.5cm}|}{The indexed attribute specification is not valid.} \\\hline 
\end{longtable}
\begin{longtable} {|p{2cm}|p{3.5cm}|p{2.5cm}|p{6.5cm}|}
	\caption{Operation Hawk.removeIndexedAttribute}
	\label{tab:hawk-removeIndexedAttribute}\\
	\hline%
	Auth: Yes & \textbf{Name} & \textbf{Type} & \textbf{Documentation} \\\hline 
	\textbf{Operation} & \camelcase{removeIndexedAttribute} & void  & Remove a indexed attribute from a Hawk instance. \\\hline
	\textbf{Parameter}s & & & \\\hline
	 & name & \camelcase{String} & The name of the Hawk instance. \\\hline 
	 & spec & \camelcase{IndexedAttributeSpec} & The details of the indexed attribute to be removed. \\\hline 
	\textbf{Exception}s & & & \\\hline
	 & \camelcase{HawkInstanceNotFound} & \multicolumn{2}{p{8.5cm}|}{No Hawk instance exists with that name.} \\\hline 
	 & \camelcase{HawkInstanceNotRunning} & \multicolumn{2}{p{8.5cm}|}{The selected Hawk instance is not running.} \\\hline 
\end{longtable}
\begin{longtable} {|p{2cm}|p{3.5cm}|p{2.5cm}|p{6.5cm}|}
	\caption{Operation Hawk.listIndexedAttributes}
	\label{tab:hawk-listIndexedAttributes}\\
	\hline%
	Auth: Yes & \textbf{Name} & \textbf{Type} & \textbf{Documentation} \\\hline 
	\textbf{Operation} & \camelcase{listIndexedAttributes} & \camelcase{IndexedAttributeSpec[]}  & Lists the indexed attributes of a Hawk instance. \\\hline
	\textbf{Parameter}s & & & \\\hline
	 & name & \camelcase{String} & The name of the Hawk instance. \\\hline 
	\textbf{Exception}s & & & \\\hline
	 & \camelcase{HawkInstanceNotFound} & \multicolumn{2}{p{8.5cm}|}{No Hawk instance exists with that name.} \\\hline 
	 & \camelcase{HawkInstanceNotRunning} & \multicolumn{2}{p{8.5cm}|}{The selected Hawk instance is not running.} \\\hline 
\end{longtable}
\begin{longtable} {|p{2cm}|p{3.5cm}|p{2.5cm}|p{6.5cm}|}
	\caption{Operation Hawk.getModel}
	\label{tab:hawk-getModel}\\
	\hline%
	Auth: Yes & \textbf{Name} & \textbf{Type} & \textbf{Documentation} \\\hline 
	\textbf{Operation} & \camelcase{getModel} & \camelcase{ModelElement[]}  & Returns the contents of one or more models indexed in a Hawk instance. Cross-model references are also resolved. \\\hline
	\textbf{Parameter}s & & & \\\hline
	 & name & \camelcase{String} & The name of the Hawk instance. \\\hline 
	 & repositoryUri & \camelcase{String[]} & The URI of the repository in which the model is contained. \\\hline 
	 & filePath & \camelcase{String[]} & The pattern(s) for the model file(s) in the repository. \\\hline 
	 & includeAttributes  (optional)& \camelcase{EBooleanObject} & Whether to include attributes (true) or not (false). \\\hline 
	 & includeReferences  (optional)& \camelcase{EBooleanObject} & Whether to include references (true) or not (false). \\\hline 
	 & includeNodeIDs  (optional)& \camelcase{EBooleanObject} & Whether to include node IDs (true) or not (false). \\\hline 
	\textbf{Exception}s & & & \\\hline
	 & \camelcase{HawkInstanceNotFound} & \multicolumn{2}{p{8.5cm}|}{No Hawk instance exists with that name.} \\\hline 
	 & \camelcase{HawkInstanceNotRunning} & \multicolumn{2}{p{8.5cm}|}{The selected Hawk instance is not running.} \\\hline 
\end{longtable}
\begin{longtable} {|p{2cm}|p{3.5cm}|p{2.5cm}|p{6.5cm}|}
	\caption{Operation Hawk.getRootElements}
	\label{tab:hawk-getRootElements}\\
	\hline%
	Auth: Yes & \textbf{Name} & \textbf{Type} & \textbf{Documentation} \\\hline 
	\textbf{Operation} & \camelcase{getRootElements} & \camelcase{ModelElement[]}  & Returns the root objects of one or more models indexed in a Hawk instance. \\\hline
	\textbf{Parameter}s & & & \\\hline
	 & name & \camelcase{String} & The name of the Hawk instance. \\\hline 
	 & repositoryUri & \camelcase{String[]} & The URI of the repository in which the model is contained. \\\hline 
	 & filePath & \camelcase{String[]} & The pattern(s) for the model file(s) in the repository. \\\hline 
	 & includeAttributes  (optional)& \camelcase{EBooleanObject} & Whether to include attributes (true) or not (false). \\\hline 
	 & includeReferences  (optional)& \camelcase{EBooleanObject} & Whether to include references (true) or not (false). \\\hline 
\end{longtable}
\begin{longtable} {|p{2cm}|p{3.5cm}|p{2.5cm}|p{6.5cm}|}
	\caption{Operation Hawk.watchModelChanges}
	\label{tab:hawk-watchModelChanges}\\
	\hline%
	Auth: Yes & \textbf{Name} & \textbf{Type} & \textbf{Documentation} \\\hline 
	\textbf{Operation} & \camelcase{watchModelChanges} & \camelcase{Subscription}  & Returns subscription details to a queue of HawkChangeEvents with notifications about changes to a set of indexed models. \\\hline
	\textbf{Parameter}s & & & \\\hline
	 & name & \camelcase{String} & The name of the Hawk instance. \\\hline 
	 & repositoryUri & \camelcase{String} & The URI of the repository in which the model is contained. \\\hline 
	 & filePath & \camelcase{String[]} & The pattern(s) for the model file(s) in the repository. \\\hline 
	 & clientID & \camelcase{String} & Unique client ID (used as suffix for the queue name). \\\hline 
	 & durableEvents & \camelcase{SubscriptionDurability} & Durability of the subscription. \\\hline 
	\textbf{Exception}s & & & \\\hline
	 & \camelcase{HawkInstanceNotFound} & \multicolumn{2}{p{8.5cm}|}{No Hawk instance exists with that name.} \\\hline 
	 & \camelcase{HawkInstanceNotRunning} & \multicolumn{2}{p{8.5cm}|}{The selected Hawk instance is not running.} \\\hline 
\end{longtable}

\subsubsection{OfflineCollaboration}
\label{sec:OfflineCollaboration}
The following service operations expose the capabilities of the offline collaboration framework
developed in Work Package 4. The framework is discussed in detail in D4.3.

\begin{longtable} {|p{2cm}|p{3.5cm}|p{2.5cm}|p{6.5cm}|}
	\caption{Operation OfflineCollaboration.checkout}
	\label{tab:offlineCollaboration-checkout}\\
	\hline%
	Auth: Yes & \textbf{Name} & \textbf{Type} & \textbf{Documentation} \\\hline 
	\textbf{Operation} & \camelcase{checkout} & \camelcase{CollaborationResource[]}  & Performs the checkout operation. \\\hline
	\textbf{Parameter}s & & & \\\hline
	 & credentials & \camelcase{Credentials} & The credentials of the user in the underlying VCS. \\\hline 
	 & resources & \camelcase{CollaborationResourceReference[]} & The references to the required resources. \\\hline 
	\textbf{Exception}s & & & \\\hline
	 & \camelcase{VCSAuthenticationFailed} & \multicolumn{2}{p{8.5cm}|}{The client failed to prove its identity in the VCS.} \\\hline 
	 & \camelcase{VCSAuthorizationFailed} & \multicolumn{2}{p{8.5cm}|}{The client does not have the required permissions in the VCS to perform the operation.} \\\hline 
	 & \camelcase{CollaborationResourceNotFound} & \multicolumn{2}{p{8.5cm}|}{The resource does not exist in the VCS.} \\\hline 
\end{longtable}
\begin{longtable} {|p{2cm}|p{3.5cm}|p{2.5cm}|p{6.5cm}|}
	\caption{Operation OfflineCollaboration.commit}
	\label{tab:offlineCollaboration-commit}\\
	\hline%
	Auth: Yes & \textbf{Name} & \textbf{Type} & \textbf{Documentation} \\\hline 
	\textbf{Operation} & \camelcase{commit} & void  & Performs the commit operation. \\\hline
	\textbf{Parameter}s & & & \\\hline
	 & credentials & \camelcase{Credentials} & The credentials of the user in the underlying VCS. \\\hline 
	 & resources & \camelcase{CollaborationResourceReference[]} & The references to the required resources. \\\hline 
	\textbf{Exception}s & & & \\\hline
	 & \camelcase{VCSAuthenticationFailed} & \multicolumn{2}{p{8.5cm}|}{The client failed to prove its identity in the VCS.} \\\hline 
	 & \camelcase{VCSAuthorizationFailed} & \multicolumn{2}{p{8.5cm}|}{The client does not have the required permissions in the VCS to perform the operation.} \\\hline 
	 & \camelcase{CollaborationResourceNotFound} & \multicolumn{2}{p{8.5cm}|}{The resource does not exist in the VCS.} \\\hline 
	 & \camelcase{CollaborationResourceLocked} & \multicolumn{2}{p{8.5cm}|}{The resource is currently locked for collaboration.} \\\hline 
\end{longtable}
\begin{longtable} {|p{2cm}|p{3.5cm}|p{2.5cm}|p{6.5cm}|}
	\caption{Operation OfflineCollaboration.pull}
	\label{tab:offlineCollaboration-pull}\\
	\hline%
	Auth: Yes & \textbf{Name} & \textbf{Type} & \textbf{Documentation} \\\hline 
	\textbf{Operation} & \camelcase{pull} & \camelcase{CollaborationResource[]}  & Performs the pull operation. \\\hline
	\textbf{Parameter}s & & & \\\hline
	 & credentials & \camelcase{Credentials} & The credentials of the user in the underlying VCS. \\\hline 
	 & resources & \camelcase{CollaborationResourceReference[]} & The references to the required resources. \\\hline 
	 & operationModel & \camelcase{OperationModel} & The operations executed on the client. \\\hline 
	\textbf{Exception}s & & & \\\hline
	 & \camelcase{VCSAuthenticationFailed} & \multicolumn{2}{p{8.5cm}|}{The client failed to prove its identity in the VCS.} \\\hline 
	 & \camelcase{VCSAuthorizationFailed} & \multicolumn{2}{p{8.5cm}|}{The client does not have the required permissions in the VCS to perform the operation.} \\\hline 
	 & \camelcase{CollaborationResourceNotFound} & \multicolumn{2}{p{8.5cm}|}{The resource does not exist in the VCS.} \\\hline 
	 & \camelcase{MergeRequired} & \multicolumn{2}{p{8.5cm}|}{The operation requires a merge before it can be retried.} \\\hline 
\end{longtable}
\begin{longtable} {|p{2cm}|p{3.5cm}|p{2.5cm}|p{6.5cm}|}
	\caption{Operation OfflineCollaboration.publishLockDefinition}
	\label{tab:offlineCollaboration-publishLockDefinition}\\
	\hline%
	Auth: Yes & \textbf{Name} & \textbf{Type} & \textbf{Documentation} \\\hline 
	\textbf{Operation} & \camelcase{publishLockDefinition} & void  & Publishes a lock definition. \\\hline
	\textbf{Parameter}s & & & \\\hline
	 & credentials & \camelcase{Credentials} & The credentials of the user in the underlying VCS. \\\hline 
	 & specification & \camelcase{CollaborationLockQuerySpec} & The lock query specification. \\\hline 
	\textbf{Exception}s & & & \\\hline
	 & \camelcase{VCSAuthenticationFailed} & \multicolumn{2}{p{8.5cm}|}{The client failed to prove its identity in the VCS.} \\\hline 
	 & \camelcase{VCSAuthorizationFailed} & \multicolumn{2}{p{8.5cm}|}{The client does not have the required permissions in the VCS to perform the operation.} \\\hline 
	 & \camelcase{InvalidCollaborationLockQuerySpec} & \multicolumn{2}{p{8.5cm}|}{The lock query specification is not valid.} \\\hline 
\end{longtable}
\begin{longtable} {|p{2cm}|p{3.5cm}|p{2.5cm}|p{6.5cm}|}
	\caption{Operation OfflineCollaboration.unpublishLockDefinition}
	\label{tab:offlineCollaboration-unpublishLockDefinition}\\
	\hline%
	Auth: Yes & \textbf{Name} & \textbf{Type} & \textbf{Documentation} \\\hline 
	\textbf{Operation} & \camelcase{unpublishLockDefinition} & void  & Unpublish a lock definition. \\\hline
	\textbf{Parameter}s & & & \\\hline
	 & credentials & \camelcase{Credentials} & The credentials of the user in the underlying VCS. \\\hline 
	 & specification & \camelcase{CollaborationLockQuerySpec} & The lock query specification. \\\hline 
	\textbf{Exception}s & & & \\\hline
	 & \camelcase{VCSAuthenticationFailed} & \multicolumn{2}{p{8.5cm}|}{The client failed to prove its identity in the VCS.} \\\hline 
	 & \camelcase{VCSAuthorizationFailed} & \multicolumn{2}{p{8.5cm}|}{The client does not have the required permissions in the VCS to perform the operation.} \\\hline 
	 & \camelcase{InvalidCollaborationLockQuerySpec} & \multicolumn{2}{p{8.5cm}|}{The lock query specification is not valid.} \\\hline 
	 & \camelcase{CollaborationLockQueryNotFound} & \multicolumn{2}{p{8.5cm}|}{No matching lock exists.} \\\hline 
\end{longtable}
\begin{longtable} {|p{2cm}|p{3.5cm}|p{2.5cm}|p{6.5cm}|}
	\caption{Operation OfflineCollaboration.lock}
	\label{tab:offlineCollaboration-lock}\\
	\hline%
	Auth: Yes & \textbf{Name} & \textbf{Type} & \textbf{Documentation} \\\hline 
	\textbf{Operation} & \camelcase{lock} & void  & Locks the pattern with the given bindings. \\\hline
	\textbf{Parameter}s & & & \\\hline
	 & credentials & \camelcase{Credentials} & The credentials of the user in the underlying VCS. \\\hline 
	 & specification & \camelcase{CollaborationQueryInvocationSpecification} & The lock specification with pattern and its bindings. \\\hline 
	\textbf{Exception}s & & & \\\hline
	 & \camelcase{VCSAuthenticationFailed} & \multicolumn{2}{p{8.5cm}|}{The client failed to prove its identity in the VCS.} \\\hline 
	 & \camelcase{VCSAuthorizationFailed} & \multicolumn{2}{p{8.5cm}|}{The client does not have the required permissions in the VCS to perform the operation.} \\\hline 
	 & \camelcase{InvalidCollaborationLockQuerySpec} & \multicolumn{2}{p{8.5cm}|}{The lock query specification is not valid.} \\\hline 
	 & \camelcase{CollaborationResourceLocked} & \multicolumn{2}{p{8.5cm}|}{The resource is currently locked for collaboration.} \\\hline 
\end{longtable}
\begin{longtable} {|p{2cm}|p{3.5cm}|p{2.5cm}|p{6.5cm}|}
	\caption{Operation OfflineCollaboration.unlock}
	\label{tab:offlineCollaboration-unlock}\\
	\hline%
	Auth: Yes & \textbf{Name} & \textbf{Type} & \textbf{Documentation} \\\hline 
	\textbf{Operation} & \camelcase{unlock} & void  & Unlocks the pattern with the given bindings. \\\hline
	\textbf{Parameter}s & & & \\\hline
	 & credentials & \camelcase{Credentials} & The credentials of the user in the underlying VCS. \\\hline 
	 & specification & \camelcase{CollaborationQueryInvocationSpecification} & The lock specification with pattern and its bindings. \\\hline 
	\textbf{Exception}s & & & \\\hline
	 & \camelcase{VCSAuthenticationFailed} & \multicolumn{2}{p{8.5cm}|}{The client failed to prove its identity in the VCS.} \\\hline 
	 & \camelcase{VCSAuthorizationFailed} & \multicolumn{2}{p{8.5cm}|}{The client does not have the required permissions in the VCS to perform the operation.} \\\hline 
	 & \camelcase{InvalidCollaborationLockQuerySpec} & \multicolumn{2}{p{8.5cm}|}{The lock query specification is not valid.} \\\hline 
	 & \camelcase{CollaborationLockQueryNotFound} & \multicolumn{2}{p{8.5cm}|}{No matching lock exists.} \\\hline 
\end{longtable}

\subsubsection{CloudATL}
\label{sec:CloudATL}
The following service operations expose the capabilities of the cloud-enabled
version of the ATL transformation language which is currently under development and
will be presented in M24 in D3.3.

\begin{longtable} {|p{2cm}|p{3.5cm}|p{2.5cm}|p{6.5cm}|}
	\caption{Operation CloudATL.launch}
	\label{tab:cloudATL-launch}\\
	\hline%
	Auth: Yes & \textbf{Name} & \textbf{Type} & \textbf{Documentation} \\\hline 
	\textbf{Operation} & \camelcase{launch} & \camelcase{String}  & Invokes a cloud-based transformation in a batch non-blocking mode.
	                                                                			Returns a token that can be used to check the status of the transformation. \\\hline
	\textbf{Parameter}s & & & \\\hline
	 & transformation & \camelcase{String} & The ATL source-code of the transformation. \\\hline 
	 & source & \camelcase{ModelSpec} & The input models of the transformation. \\\hline 
	 & target & \camelcase{ModelSpec} & The target models of the transformation. \\\hline 
	\textbf{Exception}s & & & \\\hline
	 & \camelcase{InvalidTransformation} & \multicolumn{2}{p{8.5cm}|}{The transformation is not valid: it is unparsable or inconsistent.} \\\hline 
	 & \camelcase{InvalidModelSpec} & \multicolumn{2}{p{8.5cm}|}{The model specification is not valid: the model or the metamodels are inaccessible or invalid.} \\\hline 
\end{longtable}
\begin{longtable} {|p{2cm}|p{3.5cm}|p{2.5cm}|p{6.5cm}|}
	\caption{Operation CloudATL.getJobs}
	\label{tab:cloudATL-getJobs}\\
	\hline%
	Auth: Yes & \textbf{Name} & \textbf{Type} & \textbf{Documentation} \\\hline 
	\textbf{Operation} & \camelcase{getJobs} & \camelcase{String[]}  & Lists the ids of the transformation jobs tracked by this server. \\\hline
	\textbf{Parameter}s & & & \\\hline
\end{longtable}
\begin{longtable} {|p{2cm}|p{3.5cm}|p{2.5cm}|p{6.5cm}|}
	\caption{Operation CloudATL.getStatus}
	\label{tab:cloudATL-getStatus}\\
	\hline%
	Auth: Yes & \textbf{Name} & \textbf{Type} & \textbf{Documentation} \\\hline 
	\textbf{Operation} & \camelcase{getStatus} & \camelcase{TransformationStatus}  & Returns the status of a previously invoked transformation. \\\hline
	\textbf{Parameter}s & & & \\\hline
	 & token & \camelcase{String} & A valid token returned by a previous call to launch(). \\\hline 
	\textbf{Exception}s & & & \\\hline
	 & \camelcase{TransformationTokenNotFound} & \multicolumn{2}{p{8.5cm}|}{The specified transformation token does not exist within the invokved MONDO instance.} \\\hline 
\end{longtable}
\begin{longtable} {|p{2cm}|p{3.5cm}|p{2.5cm}|p{6.5cm}|}
	\caption{Operation CloudATL.kill}
	\label{tab:cloudATL-kill}\\
	\hline%
	Auth: Yes & \textbf{Name} & \textbf{Type} & \textbf{Documentation} \\\hline 
	\textbf{Operation} & \camelcase{kill} & void  & Kills a previously invoked transformation. \\\hline
	\textbf{Parameter}s & & & \\\hline
	 & token & \camelcase{String} & A valid token returned by a previous call to launch(). \\\hline 
	\textbf{Exception}s & & & \\\hline
	 & \camelcase{TransformationTokenNotFound} & \multicolumn{2}{p{8.5cm}|}{The specified transformation token does not exist within the invokved MONDO instance.} \\\hline 
\end{longtable}


\subsection{Entities}
\label{sec:entities}
\subsubsection{AttributeSlot}

Represents a slot that can store the value(s) of an attribute of a model element.

\emph{Inherits from}: Slot.

\begin{longtable} {|p{4cm}|p{4cm}|p{7.25cm}|}
	\caption{Entity AttributeSlot}
	\label{tab:attributeSlot}\\
	\hline%
	\textbf{Field} & \textbf{Type} & \textbf{Documentation} \\\hline
		  name (\nohyphens{inherited}) & \camelcase{String} & The name of the model element property the value of which is stored in this slot. \\\hline
		  value & \camelcase{SlotValue} (optional) & Value of the slot (if set). \\\hline
   \multicolumn{3}{|p{.95\linewidth}|}{\textbf{Used in:} ModelElement} \\\hline
\end{longtable}

\subsubsection{CollaborationGitResourceReference}

Reference to a resource in a Git repository.

\emph{Inherits from}: CollaborationResourceReference.

\begin{longtable} {|p{4cm}|p{4cm}|p{7.25cm}|}
	\caption{Entity CollaborationGitResourceReference}
	\label{tab:collaborationGitResourceReference}\\
	\hline%
	\textbf{Field} & \textbf{Type} & \textbf{Documentation} \\\hline
		  branch & \camelcase{String} & The name of the Git branch to which new commits should be pushed. \\\hline
		  commit & \camelcase{String} & The SHA1 identifier of the commit with the resource. \\\hline
		  repositoryUri (\nohyphens{inherited}) & \camelcase{String} & The URI of the repository containing the resource. \\\hline
   
\end{longtable}

\subsubsection{CollaborationLockQuerySpec}

Specification of a lock on a set of model elements, without bindings.

\begin{longtable} {|p{4cm}|p{4cm}|p{7.25cm}|}
	\caption{Entity CollaborationLockQuerySpec}
	\label{tab:collaborationLockQuerySpec}\\
	\hline%
	\textbf{Field} & \textbf{Type} & \textbf{Documentation} \\\hline
		  patternFQN & \camelcase{String} & Fully qualified name of the pre-existing query. \\\hline
   \multicolumn{3}{|p{.95\linewidth}|}{\textbf{Used in:} OfflineCollaboration.publishLockDefinition, OfflineCollaboration.unpublishLockDefinition} \\\hline
\end{longtable}

\subsubsection{CollaborationQueryBinding}

Name/value binding used within a lock pattern.

\begin{longtable} {|p{4cm}|p{4cm}|p{7.25cm}|}
	\caption{Entity CollaborationQueryBinding}
	\label{tab:collaborationQueryBinding}\\
	\hline%
	\textbf{Field} & \textbf{Type} & \textbf{Documentation} \\\hline
		  name & \camelcase{String} & Name of the query parameter being bound. \\\hline
		  value & \camelcase{String} & Value to be bound to the query parameter. \\\hline
   \multicolumn{3}{|p{.95\linewidth}|}{\textbf{Used in:} CollaborationQueryInvocationSpecification} \\\hline
\end{longtable}

\subsubsection{CollaborationQueryInvocationSpecification}

Specification of a lock on a set of model elements, with bindings.

\begin{longtable} {|p{4cm}|p{4cm}|p{7.25cm}|}
	\caption{Entity CollaborationQueryInvocationSpecification}
	\label{tab:collaborationQueryInvocationSpecification}\\
	\hline%
	\textbf{Field} & \textbf{Type} & \textbf{Documentation} \\\hline
		  bindings & \camelcase{CollaborationQueryBinding[]} & Name/value bindings to be provided to the query. \\\hline
		  patternFQN & \camelcase{String} & Fully qualified name of the pre-existing query. \\\hline
   \multicolumn{3}{|p{.95\linewidth}|}{\textbf{Used in:} OfflineCollaboration.lock, OfflineCollaboration.unlock} \\\hline
\end{longtable}

\subsubsection{CollaborationResource}

Contents of a resource in the collaboration API.

\begin{longtable} {|p{4cm}|p{4cm}|p{7.25cm}|}
	\caption{Entity CollaborationResource}
	\label{tab:collaborationResource}\\
	\hline%
	\textbf{Field} & \textbf{Type} & \textbf{Documentation} \\\hline
		  file & \camelcase{File} & File with the contents of the resource. \\\hline
   \multicolumn{3}{|p{.95\linewidth}|}{\textbf{Used in:} OfflineCollaboration.checkout, OfflineCollaboration.pull} \\\hline
\end{longtable}

\subsubsection{CollaborationResourceReference}

Base entity for resource references in the collaboration API.

\emph{Inherited by}: CollaborationGitResourceReference, CollaborationSvnResourceReference.

\begin{longtable} {|p{4cm}|p{4cm}|p{7.25cm}|}
	\caption{Entity CollaborationResourceReference}
	\label{tab:collaborationResourceReference}\\
	\hline%
	\textbf{Field} & \textbf{Type} & \textbf{Documentation} \\\hline
		  repositoryUri & \camelcase{String} & The URI of the repository containing the resource. \\\hline
   \multicolumn{3}{|p{.95\linewidth}|}{\textbf{Used in:} OfflineCollaboration.checkout, OfflineCollaboration.commit, OfflineCollaboration.pull, CollaborationResourceNotFound, CollaborationResourceLocked} \\\hline
\end{longtable}

\subsubsection{CollaborationSvnResourceReference}

Reference to a resource in a SVN repository.

\emph{Inherits from}: CollaborationResourceReference.

\begin{longtable} {|p{4cm}|p{4cm}|p{7.25cm}|}
	\caption{Entity CollaborationSvnResourceReference}
	\label{tab:collaborationSvnResourceReference}\\
	\hline%
	\textbf{Field} & \textbf{Type} & \textbf{Documentation} \\\hline
		  filePath & \camelcase{String} & The path to the resource within the SVN repository. \\\hline
		  repositoryUri (\nohyphens{inherited}) & \camelcase{String} & The URI of the repository containing the resource. \\\hline
		  revision & \camelcase{String} & The revision number containing the resource. \\\hline
   
\end{longtable}

\subsubsection{CommitItem}

Simplified entry within a commit of a repository.

\begin{longtable} {|p{4cm}|p{4cm}|p{7.25cm}|}
	\caption{Entity CommitItem}
	\label{tab:commitItem}\\
	\hline%
	\textbf{Field} & \textbf{Type} & \textbf{Documentation} \\\hline
		  path & \camelcase{String} &  \\\hline
		  repoURL & \camelcase{String} &  \\\hline
		  revision & \camelcase{String} &  \\\hline
		  type & \camelcase{CommitItemChangeType} &  \\\hline
   \multicolumn{3}{|p{.95\linewidth}|}{\textbf{Used in:} HawkModelElementAdditionEvent, HawkModelElementRemovalEvent, HawkAttributeUpdateEvent, HawkAttributeRemovalEvent, HawkReferenceAdditionEvent, HawkReferenceRemovalEvent, HawkFileAdditionEvent, HawkFileRemovalEvent} \\\hline
\end{longtable}

\subsubsection{ContainerSlot}

Represents a slot that can store other model elements within a model element.

\emph{Inherits from}: Slot.

\begin{longtable} {|p{4cm}|p{4cm}|p{7.25cm}|}
	\caption{Entity ContainerSlot}
	\label{tab:containerSlot}\\
	\hline%
	\textbf{Field} & \textbf{Type} & \textbf{Documentation} \\\hline
		  elements & \camelcase{ModelElement[]} & Contained elements for this slot. \\\hline
		  name (\nohyphens{inherited}) & \camelcase{String} & The name of the model element property the value of which is stored in this slot. \\\hline
   \multicolumn{3}{|p{.95\linewidth}|}{\textbf{Used in:} ModelElement} \\\hline
\end{longtable}

\subsubsection{Credentials}

Credentials of the client in the target VCS.

\begin{longtable} {|p{4cm}|p{4cm}|p{7.25cm}|}
	\caption{Entity Credentials}
	\label{tab:credentials}\\
	\hline%
	\textbf{Field} & \textbf{Type} & \textbf{Documentation} \\\hline
		  password & \camelcase{String} & Password for logging into the VCS. \\\hline
		  username & \camelcase{String} & Username for logging into the VCS. \\\hline
   \multicolumn{3}{|p{.95\linewidth}|}{\textbf{Used in:} Hawk.addRepository, Hawk.updateRepositoryCredentials, OfflineCollaboration.checkout, OfflineCollaboration.commit, OfflineCollaboration.pull, OfflineCollaboration.publishLockDefinition, OfflineCollaboration.unpublishLockDefinition, OfflineCollaboration.lock, OfflineCollaboration.unlock} \\\hline
\end{longtable}

\subsubsection{DerivedAttributeSpec}

Used to configure Hawk's derived attributes (discussed in D5.3).

\begin{longtable} {|p{4cm}|p{4cm}|p{7.25cm}|}
	\caption{Entity DerivedAttributeSpec}
	\label{tab:derivedAttributeSpec}\\
	\hline%
	\textbf{Field} & \textbf{Type} & \textbf{Documentation} \\\hline
		  attributeName & \camelcase{String} & The name of the derived attribute. \\\hline
		  attributeType & \camelcase{String} (optional) & The (primitive) type of the derived attribute. \\\hline
		  derivationLanguage & \camelcase{String} (optional) & The language used to express the derivation logic. \\\hline
		  derivationLogic & \camelcase{String} (optional) & An executable expression of the derivation logic in the language above. \\\hline
		  isMany & \camelcase{boolean} (optional) & The multiplicity of the derived attribute. \\\hline
		  isOrdered & \camelcase{boolean} (optional) & A flag specifying whether the order of the values of the derived attribute is significant (only makes sense when isMany=true). \\\hline
		  isUnique & \camelcase{boolean} (optional) & A flag specifying whether the the values of the derived attribute are unique (only makes sense when isMany=true). \\\hline
		  metamodelUri & \camelcase{String} & The URI of the metamodel to which the derived attribute belongs. \\\hline
		  typeName & \camelcase{String} & The name of the type to which the derived attribute belongs. \\\hline
   \multicolumn{3}{|p{.95\linewidth}|}{\textbf{Used in:} Hawk.addDerivedAttribute, Hawk.removeDerivedAttribute, Hawk.listDerivedAttributes} \\\hline
\end{longtable}

\subsubsection{File}

A file to be sent through the network.

\begin{longtable} {|p{4cm}|p{4cm}|p{7.25cm}|}
	\caption{Entity File}
	\label{tab:file}\\
	\hline%
	\textbf{Field} & \textbf{Type} & \textbf{Documentation} \\\hline
		  contents & \camelcase{EByteArray} &  \\\hline
		  name & \camelcase{String} &  \\\hline
   \multicolumn{3}{|p{.95\linewidth}|}{\textbf{Used in:} Hawk.registerMetamodels, CollaborationResource} \\\hline
\end{longtable}

\subsubsection{HawkAttributeRemovalEvent}

Serialized form of an attribute removal event.

\begin{longtable} {|p{4cm}|p{4cm}|p{7.25cm}|}
	\caption{Entity HawkAttributeRemovalEvent}
	\label{tab:hawkAttributeRemovalEvent}\\
	\hline%
	\textbf{Field} & \textbf{Type} & \textbf{Documentation} \\\hline
		  attribute & \camelcase{String} & Name of the attribute that was removed. \\\hline
		  id & \camelcase{String} & Identifier of the model element that was changed. \\\hline
		  vcsItem & \camelcase{CommitItem} & Entry within the commit that produced the changes. \\\hline
   \multicolumn{3}{|p{.95\linewidth}|}{\textbf{Used in:} HawkChangeEvent} \\\hline
\end{longtable}

\subsubsection{HawkAttributeUpdateEvent}

Serialized form of an attribute update event.

\begin{longtable} {|p{4cm}|p{4cm}|p{7.25cm}|}
	\caption{Entity HawkAttributeUpdateEvent}
	\label{tab:hawkAttributeUpdateEvent}\\
	\hline%
	\textbf{Field} & \textbf{Type} & \textbf{Documentation} \\\hline
		  attribute & \camelcase{String} & Name of the attribute that was changed. \\\hline
		  id & \camelcase{String} & Identifier of the model element that was changed. \\\hline
		  value & \camelcase{SlotValue} & New value for the attribute. \\\hline
		  vcsItem & \camelcase{CommitItem} & Entry within the commit that produced the changes. \\\hline
   \multicolumn{3}{|p{.95\linewidth}|}{\textbf{Used in:} HawkChangeEvent} \\\hline
\end{longtable}

\subsubsection{HawkChangeEvent}

Serialized form of a change in the indexed models of a Hawk instance.

\begin{longtable} {|p{4cm}|p{4cm}|p{7.25cm}|}
	\caption{Entity HawkChangeEvent}
	\label{tab:hawkChangeEvent}\\
	\hline%
	\textbf{Field} & \textbf{Type} & \textbf{Documentation} \\\hline
		  fileAddition & \camelcase{HawkFileAdditionEvent} & A file was added. \\\hline
		  fileRemoval & \camelcase{HawkFileRemovalEvent} & A file was removed. \\\hline
		  modelElementAddition & \camelcase{HawkModelElementAdditionEvent} & A model element was added. \\\hline
		  modelElementAttributeRemoval & \camelcase{HawkAttributeRemovalEvent} & An attribute was removed. \\\hline
		  modelElementAttributeUpdate & \camelcase{HawkAttributeUpdateEvent} & An attribute was updated. \\\hline
		  modelElementRemoval & \camelcase{HawkModelElementRemovalEvent} & A model element was removed. \\\hline
		  referenceAddition & \camelcase{HawkReferenceAdditionEvent} & A reference was added. \\\hline
		  referenceRemoval & \camelcase{HawkReferenceRemovalEvent} & A reference was removed. \\\hline
		  syncEnd & \camelcase{HawkSynchronizationEndEvent} & Synchronization ended. \\\hline
		  syncStart & \camelcase{HawkSynchronizationStartEvent} & Synchronization started. \\\hline
   
\end{longtable}

\subsubsection{HawkFileAdditionEvent}

Serialized form of a file addition event.

\begin{longtable} {|p{4cm}|p{4cm}|p{7.25cm}|}
	\caption{Entity HawkFileAdditionEvent}
	\label{tab:hawkFileAdditionEvent}\\
	\hline%
	\textbf{Field} & \textbf{Type} & \textbf{Documentation} \\\hline
		  vcsItem & \camelcase{CommitItem} & Reference to file that was added, including VCS metadata. \\\hline
   \multicolumn{3}{|p{.95\linewidth}|}{\textbf{Used in:} HawkChangeEvent} \\\hline
\end{longtable}

\subsubsection{HawkFileRemovalEvent}

A file was removed.

\begin{longtable} {|p{4cm}|p{4cm}|p{7.25cm}|}
	\caption{Entity HawkFileRemovalEvent}
	\label{tab:hawkFileRemovalEvent}\\
	\hline%
	\textbf{Field} & \textbf{Type} & \textbf{Documentation} \\\hline
		  vcsItem & \camelcase{CommitItem} & Reference to file that was removed, including VCS metadata. \\\hline
   \multicolumn{3}{|p{.95\linewidth}|}{\textbf{Used in:} HawkChangeEvent} \\\hline
\end{longtable}

\subsubsection{HawkInstance}

Status of a Hawk instance.

\begin{longtable} {|p{4cm}|p{4cm}|p{7.25cm}|}
	\caption{Entity HawkInstance}
	\label{tab:hawkInstance}\\
	\hline%
	\textbf{Field} & \textbf{Type} & \textbf{Documentation} \\\hline
		  name & \camelcase{String} & The name of the instance. \\\hline
		  running & \camelcase{boolean} & Whether the instance is running or not. \\\hline
   \multicolumn{3}{|p{.95\linewidth}|}{\textbf{Used in:} Hawk.listInstances} \\\hline
\end{longtable}

\subsubsection{HawkModelElementAdditionEvent}

Serialized form of a model element addition event.

\begin{longtable} {|p{4cm}|p{4cm}|p{7.25cm}|}
	\caption{Entity HawkModelElementAdditionEvent}
	\label{tab:hawkModelElementAdditionEvent}\\
	\hline%
	\textbf{Field} & \textbf{Type} & \textbf{Documentation} \\\hline
		  id & \camelcase{String} & Identifier of the model element that was added. \\\hline
		  metamodelURI & \camelcase{String} & Metamodel URI of the type of the model element. \\\hline
		  typeName & \camelcase{String} & Name of the type of the model element. \\\hline
		  vcsItem & \camelcase{CommitItem} & Entry within the commit that produced the changes. \\\hline
   \multicolumn{3}{|p{.95\linewidth}|}{\textbf{Used in:} HawkChangeEvent} \\\hline
\end{longtable}

\subsubsection{HawkModelElementRemovalEvent}

Serialized form of a model element removal event.

\begin{longtable} {|p{4cm}|p{4cm}|p{7.25cm}|}
	\caption{Entity HawkModelElementRemovalEvent}
	\label{tab:hawkModelElementRemovalEvent}\\
	\hline%
	\textbf{Field} & \textbf{Type} & \textbf{Documentation} \\\hline
		  id & \camelcase{String} & Identifier of the model element that was removed. \\\hline
		  vcsItem & \camelcase{CommitItem} & Entry within the commit that produced the changes. \\\hline
   \multicolumn{3}{|p{.95\linewidth}|}{\textbf{Used in:} HawkChangeEvent} \\\hline
\end{longtable}

\subsubsection{HawkReferenceAdditionEvent}

Serialized form of a reference addition event.

\begin{longtable} {|p{4cm}|p{4cm}|p{7.25cm}|}
	\caption{Entity HawkReferenceAdditionEvent}
	\label{tab:hawkReferenceAdditionEvent}\\
	\hline%
	\textbf{Field} & \textbf{Type} & \textbf{Documentation} \\\hline
		  refName & \camelcase{String} & Name of the reference that was added. \\\hline
		  sourceId & \camelcase{String} & Identifier of the source model element. \\\hline
		  targetId & \camelcase{String} & Identifier of the target model element. \\\hline
		  vcsItem & \camelcase{CommitItem} & Entry within the commit that produced the changes. \\\hline
   \multicolumn{3}{|p{.95\linewidth}|}{\textbf{Used in:} HawkChangeEvent} \\\hline
\end{longtable}

\subsubsection{HawkReferenceRemovalEvent}

Serialized form of a reference removal event.

\begin{longtable} {|p{4cm}|p{4cm}|p{7.25cm}|}
	\caption{Entity HawkReferenceRemovalEvent}
	\label{tab:hawkReferenceRemovalEvent}\\
	\hline%
	\textbf{Field} & \textbf{Type} & \textbf{Documentation} \\\hline
		  refName & \camelcase{String} & Name of the reference that was removed. \\\hline
		  sourceId & \camelcase{String} & Identifier of the source model element. \\\hline
		  targetId & \camelcase{String} & Identifier of the target model element. \\\hline
		  vcsItem & \camelcase{CommitItem} & Entry within the commit that produced the changes. \\\hline
   \multicolumn{3}{|p{.95\linewidth}|}{\textbf{Used in:} HawkChangeEvent} \\\hline
\end{longtable}

\subsubsection{HawkSynchronizationEndEvent}

Serialized form of a sync end event.

\begin{longtable} {|p{4cm}|p{4cm}|p{7.25cm}|}
	\caption{Entity HawkSynchronizationEndEvent}
	\label{tab:hawkSynchronizationEndEvent}\\
	\hline%
	\textbf{Field} & \textbf{Type} & \textbf{Documentation} \\\hline
		  timestampNanos & \camelcase{long} & Local timestamp, measured in nanoseconds. Only meant to be used to compute synchronization cost. \\\hline
   \multicolumn{3}{|p{.95\linewidth}|}{\textbf{Used in:} HawkChangeEvent} \\\hline
\end{longtable}

\subsubsection{HawkSynchronizationStartEvent}

Serialized form of a sync start event.

\begin{longtable} {|p{4cm}|p{4cm}|p{7.25cm}|}
	\caption{Entity HawkSynchronizationStartEvent}
	\label{tab:hawkSynchronizationStartEvent}\\
	\hline%
	\textbf{Field} & \textbf{Type} & \textbf{Documentation} \\\hline
		  timestampNanos & \camelcase{long} & Local timestamp, measured in nanoseconds. Only meant to be used to compute synchronization cost. \\\hline
   \multicolumn{3}{|p{.95\linewidth}|}{\textbf{Used in:} HawkChangeEvent} \\\hline
\end{longtable}

\subsubsection{IndexedAttributeSpec}

Used to configure Hawk's indexed attributes (discussed in D5.3).

\begin{longtable} {|p{4cm}|p{4cm}|p{7.25cm}|}
	\caption{Entity IndexedAttributeSpec}
	\label{tab:indexedAttributeSpec}\\
	\hline%
	\textbf{Field} & \textbf{Type} & \textbf{Documentation} \\\hline
		  attributeName & \camelcase{String} & The name of the indexed attribute. \\\hline
		  metamodelUri & \camelcase{String} & The URI of the metamodel to which the indexed attribute belongs. \\\hline
		  typeName & \camelcase{String} & The name of the type to which the indexed attribute belongs. \\\hline
   \multicolumn{3}{|p{.95\linewidth}|}{\textbf{Used in:} Hawk.addIndexedAttribute, Hawk.removeIndexedAttribute, Hawk.listIndexedAttributes} \\\hline
\end{longtable}

\subsubsection{MixedReference}

Represents a reference to a model element: it can be an identifier or a position.
Only used when the same ReferenceSlot has both identifier-based and position-based references.
This may be the case if we are retrieving a subset of the model which has references
between its elements and with elements outside the subset at the same time.

\begin{longtable} {|p{4cm}|p{4cm}|p{7.25cm}|}
	\caption{Entity MixedReference}
	\label{tab:mixedReference}\\
	\hline%
	\textbf{Field} & \textbf{Type} & \textbf{Documentation} \\\hline
		  id & \camelcase{String} & Identifier-based reference to a model element. \\\hline
		  position & \camelcase{int} & Position-based reference to a model element. \\\hline
   \multicolumn{3}{|p{.95\linewidth}|}{\textbf{Used in:} ReferenceSlot} \\\hline
\end{longtable}

\subsubsection{ModelElement}

Represents a model element.

\begin{longtable} {|p{4cm}|p{4cm}|p{7.25cm}|}
	\caption{Entity ModelElement}
	\label{tab:modelElement}\\
	\hline%
	\textbf{Field} & \textbf{Type} & \textbf{Documentation} \\\hline
		  attributes & \camelcase{AttributeSlot[]} (optional) & Slots holding the values of the model element's attributes, if any have been set. \\\hline
		  containers & \camelcase{ContainerSlot[]} (optional) & Slots holding contained model elements, if any have been set. \\\hline
		  file & \camelcase{String} (optional) & Name of the file to which the element belongs (not set if equal to that of the previous model element). \\\hline
		  id & \camelcase{String} (optional) & Unique ID of the model element (not set if using position-based references). \\\hline
		  metamodelUri & \camelcase{String} (optional) & URI of the metamodel to which the type of the element belongs (not set if equal to that of the previous model element). \\\hline
		  references & \camelcase{ReferenceSlot[]} (optional) & Slots holding the values of the model element's references, if any have been set. \\\hline
		  repositoryURL & \camelcase{String} (optional) & URI of the repository to which the element belongs (not set if equal to that of the previous model element). \\\hline
		  typeName & \camelcase{String} (optional) & Name of the type that the model element is an instance of (not set if equal to that of the previous model element). \\\hline
   \multicolumn{3}{|p{.95\linewidth}|}{\textbf{Used in:} Hawk.resolveProxies, Hawk.getModel, Hawk.getRootElements, ContainerSlot, QueryResult} \\\hline
\end{longtable}

\subsubsection{ModelElementType}

Represents a type of model element.

\begin{longtable} {|p{4cm}|p{4cm}|p{7.25cm}|}
	\caption{Entity ModelElementType}
	\label{tab:modelElementType}\\
	\hline%
	\textbf{Field} & \textbf{Type} & \textbf{Documentation} \\\hline
		  attributes & \camelcase{SlotMetadata[]} & Metadata for the attribute slots. \\\hline
		  id & \camelcase{String} & Unique ID of the model element type. \\\hline
		  metamodelUri & \camelcase{String} & URI of the metamodel to which the type belongs. \\\hline
		  references & \camelcase{SlotMetadata[]} & Metadata for the reference slots. \\\hline
		  typeName & \camelcase{String} & Name of the type. \\\hline
   \multicolumn{3}{|p{.95\linewidth}|}{\textbf{Used in:} QueryResult} \\\hline
\end{longtable}

\subsubsection{ModelSpec}

Captures information about source/target models of ATL transformations.

\begin{longtable} {|p{4cm}|p{4cm}|p{7.25cm}|}
	\caption{Entity ModelSpec}
	\label{tab:modelSpec}\\
	\hline%
	\textbf{Field} & \textbf{Type} & \textbf{Documentation} \\\hline
		  metamodelUris & \camelcase{String[]} & The URIs of the metamodels to which elements of the model conform. \\\hline
		  uri & \camelcase{String} & The URI from which the model will be loaded or to which it will be persisted. \\\hline
   \multicolumn{3}{|p{.95\linewidth}|}{\textbf{Used in:} CloudATL.launch, InvalidModelSpec} \\\hline
\end{longtable}

\subsubsection{OperationModel}

Operational trace that encodes the operations carried out by the user locally over the models.
Its internal schema is presented in D4.3 and will be refined in D4.2.

\begin{longtable} {|p{4cm}|p{4cm}|p{7.25cm}|}
	\caption{Entity OperationModel}
	\label{tab:operationModel}\\
	\hline%
	\textbf{Field} & \textbf{Type} & \textbf{Documentation} \\\hline
	    --- & --- & --- \\\hline
   \multicolumn{3}{|p{.95\linewidth}|}{\textbf{Used in:} OfflineCollaboration.pull} \\\hline
\end{longtable}

\subsubsection{QueryResult}

Union type for a scalar value or a reference to a model element. Useful for heterogeneous collections.

\emph{Inherits from}: Value.

\begin{longtable} {|p{4cm}|p{4cm}|p{7.25cm}|}
	\caption{Entity QueryResult}
	\label{tab:queryResult}\\
	\hline%
	\textbf{Field} & \textbf{Type} & \textbf{Documentation} \\\hline
		  vBoolean (\nohyphens{inherited}) & \camelcase{boolean} &  \\\hline
		  vByte (\nohyphens{inherited}) & \camelcase{EByte} &  \\\hline
		  vDouble (\nohyphens{inherited}) & \camelcase{EDouble} &  \\\hline
		  vInteger (\nohyphens{inherited}) & \camelcase{int} &  \\\hline
		  vLong (\nohyphens{inherited}) & \camelcase{long} &  \\\hline
		  vModelElement & \camelcase{ModelElement} &  \\\hline
		  vModelElementType & \camelcase{ModelElementType} &  \\\hline
		  vShort (\nohyphens{inherited}) & \camelcase{EShort} &  \\\hline
		  vString (\nohyphens{inherited}) & \camelcase{String} &  \\\hline
   \multicolumn{3}{|p{.95\linewidth}|}{\textbf{Used in:} Hawk.query} \\\hline
\end{longtable}

\subsubsection{ReferenceSlot}

Represents a slot that can store the value(s) of a reference  of a model element.
References can be expressed as positions within a result tree (using pre-order traversal)
or IDs. id, ids, position, positions and mixed are all mutually exclusive. At least one position
or one ID must be given.

\emph{Inherits from}: Slot.

\begin{longtable} {|p{4cm}|p{4cm}|p{7.25cm}|}
	\caption{Entity ReferenceSlot}
	\label{tab:referenceSlot}\\
	\hline%
	\textbf{Field} & \textbf{Type} & \textbf{Documentation} \\\hline
		  id & \camelcase{String} (optional) & Unique identifier of the referenced element (if there is only one ID based reference in this slot). \\\hline
		  ids & \camelcase{String[]} (optional) & Unique identifiers of the referenced elements (if more than one). \\\hline
		  mixed & \camelcase{MixedReference[]} (optional) & Mix of identifier- and position-bsaed references (if there is at least one position and one ID. \\\hline
		  name (\nohyphens{inherited}) & \camelcase{String} & The name of the model element property the value of which is stored in this slot. \\\hline
		  position & \camelcase{int} (optional) & Position of the referenced element (if there is only one position-based reference in this slot). \\\hline
		  positions & \camelcase{int[]} (optional) & Positions of the referenced elements (if more than one). \\\hline
   \multicolumn{3}{|p{.95\linewidth}|}{\textbf{Used in:} ModelElement} \\\hline
\end{longtable}

\subsubsection{Repository}

Entity that represents a model repository.

\begin{longtable} {|p{4cm}|p{4cm}|p{7.25cm}|}
	\caption{Entity Repository}
	\label{tab:repository}\\
	\hline%
	\textbf{Field} & \textbf{Type} & \textbf{Documentation} \\\hline
		  type & \camelcase{String} & The type of repository. \\\hline
		  uri & \camelcase{String} & The URI to the repository. \\\hline
   \multicolumn{3}{|p{.95\linewidth}|}{\textbf{Used in:} Hawk.addRepository, Hawk.listRepositories} \\\hline
\end{longtable}

\subsubsection{Slot}

Represents a slot that can store the value(s) of a property of a model element.

\emph{Inherited by}: AttributeSlot, ReferenceSlot, ContainerSlot.

\begin{longtable} {|p{4cm}|p{4cm}|p{7.25cm}|}
	\caption{Entity Slot}
	\label{tab:slot}\\
	\hline%
	\textbf{Field} & \textbf{Type} & \textbf{Documentation} \\\hline
		  name & \camelcase{String} & The name of the model element property the value of which is stored in this slot. \\\hline
   
\end{longtable}

\subsubsection{SlotMetadata}

Represents the metadata of a slot in a model element type.

\begin{longtable} {|p{4cm}|p{4cm}|p{7.25cm}|}
	\caption{Entity SlotMetadata}
	\label{tab:slotMetadata}\\
	\hline%
	\textbf{Field} & \textbf{Type} & \textbf{Documentation} \\\hline
		  isMany & \camelcase{boolean} & True if this slot holds a collection of values instead of a single value. \\\hline
		  isOrdered & \camelcase{boolean} & True if the values in this slot are ordered. \\\hline
		  isUnique & \camelcase{boolean} & True if the value of this slot must be unique within its containing model. \\\hline
		  name & \camelcase{String} & The name of the model element property that is stored in this slot. \\\hline
		  type & \camelcase{String} & The type of the values in this slot. \\\hline
   \multicolumn{3}{|p{.95\linewidth}|}{\textbf{Used in:} ModelElementType} \\\hline
\end{longtable}

\subsubsection{SlotValue}

Union type for a single scalar value or a homogeneous collection of scalar values.

\emph{Inherits from}: Value.

\begin{longtable} {|p{4cm}|p{4cm}|p{7.25cm}|}
	\caption{Entity SlotValue}
	\label{tab:slotValue}\\
	\hline%
	\textbf{Field} & \textbf{Type} & \textbf{Documentation} \\\hline
		  vBoolean (\nohyphens{inherited}) & \camelcase{boolean} &  \\\hline
		  vBooleans & \camelcase{boolean[]} &  \\\hline
		  vByte (\nohyphens{inherited}) & \camelcase{EByte} &  \\\hline
		  vBytes & \camelcase{EByteArray} &  \\\hline
		  vDouble (\nohyphens{inherited}) & \camelcase{EDouble} &  \\\hline
		  vDoubles & \camelcase{EDouble[]} &  \\\hline
		  vInteger (\nohyphens{inherited}) & \camelcase{int} &  \\\hline
		  vIntegers & \camelcase{int[]} &  \\\hline
		  vLong (\nohyphens{inherited}) & \camelcase{long} &  \\\hline
		  vLongs & \camelcase{long[]} &  \\\hline
		  vShort (\nohyphens{inherited}) & \camelcase{EShort} &  \\\hline
		  vShorts & \camelcase{EShort[]} &  \\\hline
		  vString (\nohyphens{inherited}) & \camelcase{String} &  \\\hline
		  vStrings & \camelcase{String[]} &  \\\hline
   \multicolumn{3}{|p{.95\linewidth}|}{\textbf{Used in:} HawkAttributeUpdateEvent, AttributeSlot} \\\hline
\end{longtable}

\subsubsection{Subscription}

Details about a subscription to a topic queue.

\begin{longtable} {|p{4cm}|p{4cm}|p{7.25cm}|}
	\caption{Entity Subscription}
	\label{tab:subscription}\\
	\hline%
	\textbf{Field} & \textbf{Type} & \textbf{Documentation} \\\hline
		  host & \camelcase{String} & Host name of the message queue server. \\\hline
		  port & \camelcase{int} & Port in which the message queue server is listening. \\\hline
		  queueAddress & \camelcase{String} & Address of the topic queue. \\\hline
		  queueName & \camelcase{String} & Name of the topic queue. \\\hline
   \multicolumn{3}{|p{.95\linewidth}|}{\textbf{Used in:} Hawk.watchModelChanges} \\\hline
\end{longtable}

\subsubsection{TransformationStatus}

Used to report the status of a long-running transformation by CloudATL.

\begin{longtable} {|p{4cm}|p{4cm}|p{7.25cm}|}
	\caption{Entity TransformationStatus}
	\label{tab:transformationStatus}\\
	\hline%
	\textbf{Field} & \textbf{Type} & \textbf{Documentation} \\\hline
		  elapsed & \camelcase{long} & Time passed since the start of execution. \\\hline
		  error & \camelcase{String} & Description of the error that caused the transformation to fail. \\\hline
		  state & \camelcase{TransformationState} & State of the tranformation. \\\hline
   \multicolumn{3}{|p{.95\linewidth}|}{\textbf{Used in:} CloudATL.getStatus} \\\hline
\end{longtable}

\subsubsection{UserProfile}

Minimal details about registered users.

\begin{longtable} {|p{4cm}|p{4cm}|p{7.25cm}|}
	\caption{Entity UserProfile}
	\label{tab:userProfile}\\
	\hline%
	\textbf{Field} & \textbf{Type} & \textbf{Documentation} \\\hline
		  admin & \camelcase{boolean} & Whether the user has admin rights (i.e. so that they can create new users, change the status of admin users etc). \\\hline
		  realName & \camelcase{String} & The real name of the user. \\\hline
   \multicolumn{3}{|p{.95\linewidth}|}{\textbf{Used in:} Users.createUser, Users.updateProfile} \\\hline
\end{longtable}

\subsubsection{Value}

Union type for a single scalar value.

\emph{Inherited by}: QueryResult, SlotValue.

\begin{longtable} {|p{4cm}|p{4cm}|p{7.25cm}|}
	\caption{Entity Value}
	\label{tab:value}\\
	\hline%
	\textbf{Field} & \textbf{Type} & \textbf{Documentation} \\\hline
		  vBoolean & \camelcase{boolean} &  \\\hline
		  vByte & \camelcase{EByte} &  \\\hline
		  vDouble & \camelcase{EDouble} &  \\\hline
		  vInteger & \camelcase{int} &  \\\hline
		  vLong & \camelcase{long} &  \\\hline
		  vShort & \camelcase{EShort} &  \\\hline
		  vString & \camelcase{String} &  \\\hline
   
\end{longtable}



\subsection{Enumerations}
\label{sec:enumerations}
\subsubsection{CommitItemChangeType}

\begin{longtable} {|p{4cm}|p{10.5cm}|}
	\caption{Enumeration CommitItemChangeType}
	\label{tab:commitItemChangeType}\\
	\hline%
	\textbf{Name} & \textbf{Documentation} \\\hline 
		ADDED &  \\\hline 
		DELETED &  \\\hline 
		REPLACED &  \\\hline 
		UNKNOWN &  \\\hline 
		UPDATED &  \\\hline 
	\multicolumn{2}{|p{.95\linewidth}|}{\textbf{Used in:} CommitItem} \\\hline
\end{longtable}

\subsubsection{SubscriptionDurability}

\begin{longtable} {|p{4cm}|p{10.5cm}|}
	\caption{Enumeration SubscriptionDurability}
	\label{tab:subscriptionDurability}\\
	\hline%
	\textbf{Name} & \textbf{Documentation} \\\hline 
		DEFAULT & Subscription survives client disconnections but not server restarts. \\\hline 
		DURABLE & Subscription survives client disconnections and server restarts. \\\hline 
		TEMPORARY & Subscription removed after disconnecting. \\\hline 
	\multicolumn{2}{|p{.95\linewidth}|}{\textbf{Used in:} Hawk.watchModelChanges} \\\hline
\end{longtable}

\subsubsection{TransformationState}

\begin{longtable} {|p{4cm}|p{10.5cm}|}
	\caption{Enumeration TransformationState}
	\label{tab:transformationState}\\
	\hline%
	\textbf{Name} & \textbf{Documentation} \\\hline 
		FAILED & The transformation has failed. \\\hline 
		KILLED & The transformation was interrupted by a user (i.e. killed). \\\hline 
		PREP & The transformation is in preparation. \\\hline 
		RUNNING & The transformation is running. \\\hline 
		SUCCEEDED & The transformation has completed successfully. \\\hline 
	\multicolumn{2}{|p{.95\linewidth}|}{\textbf{Used in:} TransformationStatus} \\\hline
\end{longtable}

\subsection{Exceptions}
\label{sec:exceptions}
\subsubsection{AuthenticationFailed}
The client failed to prove its identity in MONDO or did not have the necessary permissions.

\begin{longtable} {|p{4cm}|p{4cm}|p{7.25cm}|}
	\caption{Exception AuthenticationFailed}
	\label{tab:authenticationFailed}\\
	\hline%
	\textbf{Field} & \textbf{Type} & \textbf{Documentation} \\\hline
	    --- & --- & --- \\\hline
   
\end{longtable}
\subsubsection{CollaborationLockQueryNotFound}
No matching lock exists.

\begin{longtable} {|p{4cm}|p{4cm}|p{7.25cm}|}
	\caption{Exception CollaborationLockQueryNotFound}
	\label{tab:collaborationLockQueryNotFound}\\
	\hline%
	\textbf{Field} & \textbf{Type} & \textbf{Documentation} \\\hline
	    --- & --- & --- \\\hline
   \multicolumn{3}{|p{.95\linewidth}|}{\textbf{Used in:} OfflineCollaboration.unpublishLockDefinition, OfflineCollaboration.unlock} \\\hline
\end{longtable}
\subsubsection{CollaborationResourceLocked}
The resource is currently locked for collaboration.

\begin{longtable} {|p{4cm}|p{4cm}|p{7.25cm}|}
	\caption{Exception CollaborationResourceLocked}
	\label{tab:collaborationResourceLocked}\\
	\hline%
	\textbf{Field} & \textbf{Type} & \textbf{Documentation} \\\hline
		  resourceReference & \camelcase{CollaborationResourceReference} & Reference to the locked resource. \\\hline
   \multicolumn{3}{|p{.95\linewidth}|}{\textbf{Used in:} OfflineCollaboration.commit, OfflineCollaboration.lock} \\\hline
\end{longtable}
\subsubsection{CollaborationResourceNotFound}
The resource does not exist in the VCS.

\begin{longtable} {|p{4cm}|p{4cm}|p{7.25cm}|}
	\caption{Exception CollaborationResourceNotFound}
	\label{tab:collaborationResourceNotFound}\\
	\hline%
	\textbf{Field} & \textbf{Type} & \textbf{Documentation} \\\hline
		  resourceReference & \camelcase{CollaborationResourceReference} & Reference to the missing resource. \\\hline
   \multicolumn{3}{|p{.95\linewidth}|}{\textbf{Used in:} OfflineCollaboration.checkout, OfflineCollaboration.commit, OfflineCollaboration.pull} \\\hline
\end{longtable}
\subsubsection{FailedQuery}
The specified query failed to complete its execution.

\begin{longtable} {|p{4cm}|p{4cm}|p{7.25cm}|}
	\caption{Exception FailedQuery}
	\label{tab:failedQuery}\\
	\hline%
	\textbf{Field} & \textbf{Type} & \textbf{Documentation} \\\hline
		  reason & \camelcase{String} & Reason for the query failing to complete its execution. \\\hline
   \multicolumn{3}{|p{.95\linewidth}|}{\textbf{Used in:} Hawk.query} \\\hline
\end{longtable}
\subsubsection{HawkInstanceNotFound}
No Hawk instance exists with that name.

\begin{longtable} {|p{4cm}|p{4cm}|p{7.25cm}|}
	\caption{Exception HawkInstanceNotFound}
	\label{tab:hawkInstanceNotFound}\\
	\hline%
	\textbf{Field} & \textbf{Type} & \textbf{Documentation} \\\hline
	    --- & --- & --- \\\hline
   \multicolumn{3}{|p{.95\linewidth}|}{\textbf{Used in:} Hawk.removeInstance, Hawk.startInstance, Hawk.stopInstance, Hawk.syncInstance, Hawk.registerMetamodels, Hawk.unregisterMetamodels, Hawk.listMetamodels, Hawk.query, Hawk.resolveProxies, Hawk.addRepository, Hawk.removeRepository, Hawk.updateRepositoryCredentials, Hawk.listRepositories, Hawk.listFiles, Hawk.configurePolling, Hawk.addDerivedAttribute, Hawk.removeDerivedAttribute, Hawk.listDerivedAttributes, Hawk.addIndexedAttribute, Hawk.removeIndexedAttribute, Hawk.listIndexedAttributes, Hawk.getModel, Hawk.watchModelChanges} \\\hline
\end{longtable}
\subsubsection{HawkInstanceNotRunning}
The selected Hawk instance is not running.

\begin{longtable} {|p{4cm}|p{4cm}|p{7.25cm}|}
	\caption{Exception HawkInstanceNotRunning}
	\label{tab:hawkInstanceNotRunning}\\
	\hline%
	\textbf{Field} & \textbf{Type} & \textbf{Documentation} \\\hline
	    --- & --- & --- \\\hline
   \multicolumn{3}{|p{.95\linewidth}|}{\textbf{Used in:} Hawk.stopInstance, Hawk.syncInstance, Hawk.registerMetamodels, Hawk.unregisterMetamodels, Hawk.listMetamodels, Hawk.query, Hawk.resolveProxies, Hawk.addRepository, Hawk.removeRepository, Hawk.updateRepositoryCredentials, Hawk.listRepositories, Hawk.listFiles, Hawk.configurePolling, Hawk.addDerivedAttribute, Hawk.removeDerivedAttribute, Hawk.listDerivedAttributes, Hawk.addIndexedAttribute, Hawk.removeIndexedAttribute, Hawk.listIndexedAttributes, Hawk.getModel, Hawk.watchModelChanges} \\\hline
\end{longtable}
\subsubsection{InvalidCollaborationLockQuerySpec}
The lock query specification is not valid.

\begin{longtable} {|p{4cm}|p{4cm}|p{7.25cm}|}
	\caption{Exception InvalidCollaborationLockQuerySpec}
	\label{tab:invalidCollaborationLockQuerySpec}\\
	\hline%
	\textbf{Field} & \textbf{Type} & \textbf{Documentation} \\\hline
	    --- & --- & --- \\\hline
   \multicolumn{3}{|p{.95\linewidth}|}{\textbf{Used in:} OfflineCollaboration.publishLockDefinition, OfflineCollaboration.unpublishLockDefinition, OfflineCollaboration.lock, OfflineCollaboration.unlock} \\\hline
\end{longtable}
\subsubsection{InvalidDerivedAttributeSpec}
The derived attribute specification is not valid.

\begin{longtable} {|p{4cm}|p{4cm}|p{7.25cm}|}
	\caption{Exception InvalidDerivedAttributeSpec}
	\label{tab:invalidDerivedAttributeSpec}\\
	\hline%
	\textbf{Field} & \textbf{Type} & \textbf{Documentation} \\\hline
		  reason & \camelcase{String} & Reason for the spec not being valid. \\\hline
   \multicolumn{3}{|p{.95\linewidth}|}{\textbf{Used in:} Hawk.addDerivedAttribute} \\\hline
\end{longtable}
\subsubsection{InvalidIndexedAttributeSpec}
The indexed attribute specification is not valid.

\begin{longtable} {|p{4cm}|p{4cm}|p{7.25cm}|}
	\caption{Exception InvalidIndexedAttributeSpec}
	\label{tab:invalidIndexedAttributeSpec}\\
	\hline%
	\textbf{Field} & \textbf{Type} & \textbf{Documentation} \\\hline
		  reason & \camelcase{String} & Reason for the spec not being valid. \\\hline
   \multicolumn{3}{|p{.95\linewidth}|}{\textbf{Used in:} Hawk.addIndexedAttribute} \\\hline
\end{longtable}
\subsubsection{InvalidMetamodel}
The provided metamodel is not valid (e.g. unparsable or inconsistent).

\begin{longtable} {|p{4cm}|p{4cm}|p{7.25cm}|}
	\caption{Exception InvalidMetamodel}
	\label{tab:invalidMetamodel}\\
	\hline%
	\textbf{Field} & \textbf{Type} & \textbf{Documentation} \\\hline
		  reason & \camelcase{String} & Reason for the metamodel not being valid. \\\hline
   \multicolumn{3}{|p{.95\linewidth}|}{\textbf{Used in:} Hawk.registerMetamodels} \\\hline
\end{longtable}
\subsubsection{InvalidModelSpec}
The model specification is not valid: the model or the metamodels are inaccessible or invalid.

\begin{longtable} {|p{4cm}|p{4cm}|p{7.25cm}|}
	\caption{Exception InvalidModelSpec}
	\label{tab:invalidModelSpec}\\
	\hline%
	\textbf{Field} & \textbf{Type} & \textbf{Documentation} \\\hline
		  reason & \camelcase{String} & Reason for the spec not being valid. \\\hline
		  spec & \camelcase{ModelSpec} & A copy of the invalid model specification. \\\hline
   \multicolumn{3}{|p{.95\linewidth}|}{\textbf{Used in:} CloudATL.launch} \\\hline
\end{longtable}
\subsubsection{InvalidPollingConfiguration}
The polling configuration is not valid.

\begin{longtable} {|p{4cm}|p{4cm}|p{7.25cm}|}
	\caption{Exception InvalidPollingConfiguration}
	\label{tab:invalidPollingConfiguration}\\
	\hline%
	\textbf{Field} & \textbf{Type} & \textbf{Documentation} \\\hline
		  reason & \camelcase{String} & Reason for the spec not being valid. \\\hline
   \multicolumn{3}{|p{.95\linewidth}|}{\textbf{Used in:} Hawk.configurePolling} \\\hline
\end{longtable}
\subsubsection{InvalidQuery}
The specified query is not valid.

\begin{longtable} {|p{4cm}|p{4cm}|p{7.25cm}|}
	\caption{Exception InvalidQuery}
	\label{tab:invalidQuery}\\
	\hline%
	\textbf{Field} & \textbf{Type} & \textbf{Documentation} \\\hline
		  reason & \camelcase{String} & Reason for the query not being valid. \\\hline
   \multicolumn{3}{|p{.95\linewidth}|}{\textbf{Used in:} Hawk.query} \\\hline
\end{longtable}
\subsubsection{InvalidTransformation}
The transformation is not valid: it is unparsable or inconsistent.

\begin{longtable} {|p{4cm}|p{4cm}|p{7.25cm}|}
	\caption{Exception InvalidTransformation}
	\label{tab:invalidTransformation}\\
	\hline%
	\textbf{Field} & \textbf{Type} & \textbf{Documentation} \\\hline
		  location & \camelcase{String} & Location of the problem, if applicable. Usually a combination of line and column numbers. \\\hline
		  reason & \camelcase{String} & Reason for the transformation not being valid. \\\hline
   \multicolumn{3}{|p{.95\linewidth}|}{\textbf{Used in:} CloudATL.launch} \\\hline
\end{longtable}
\subsubsection{MergeRequired}
The operation requires a merge before it can be retried.

\begin{longtable} {|p{4cm}|p{4cm}|p{7.25cm}|}
	\caption{Exception MergeRequired}
	\label{tab:mergeRequired}\\
	\hline%
	\textbf{Field} & \textbf{Type} & \textbf{Documentation} \\\hline
	    --- & --- & --- \\\hline
   \multicolumn{3}{|p{.95\linewidth}|}{\textbf{Used in:} OfflineCollaboration.pull} \\\hline
\end{longtable}
\subsubsection{TransformationTokenNotFound}
The specified transformation token does not exist within the invokved MONDO instance.

\begin{longtable} {|p{4cm}|p{4cm}|p{7.25cm}|}
	\caption{Exception TransformationTokenNotFound}
	\label{tab:transformationTokenNotFound}\\
	\hline%
	\textbf{Field} & \textbf{Type} & \textbf{Documentation} \\\hline
		  token & \camelcase{String} & Transformation token which was not found within the invoked MONDO instance. \\\hline
   \multicolumn{3}{|p{.95\linewidth}|}{\textbf{Used in:} CloudATL.getStatus, CloudATL.kill} \\\hline
\end{longtable}
\subsubsection{UnknownQueryLanguage}
The specified query language is not supported by the operation.

\begin{longtable} {|p{4cm}|p{4cm}|p{7.25cm}|}
	\caption{Exception UnknownQueryLanguage}
	\label{tab:unknownQueryLanguage}\\
	\hline%
	\textbf{Field} & \textbf{Type} & \textbf{Documentation} \\\hline
	    --- & --- & --- \\\hline
   \multicolumn{3}{|p{.95\linewidth}|}{\textbf{Used in:} Hawk.query} \\\hline
\end{longtable}
\subsubsection{UnknownRepositoryType}
The specified repository type is not supported by the operation.

\begin{longtable} {|p{4cm}|p{4cm}|p{7.25cm}|}
	\caption{Exception UnknownRepositoryType}
	\label{tab:unknownRepositoryType}\\
	\hline%
	\textbf{Field} & \textbf{Type} & \textbf{Documentation} \\\hline
	    --- & --- & --- \\\hline
   \multicolumn{3}{|p{.95\linewidth}|}{\textbf{Used in:} Hawk.addRepository} \\\hline
\end{longtable}
\subsubsection{UserExists}
The specified username already exists.

\begin{longtable} {|p{4cm}|p{4cm}|p{7.25cm}|}
	\caption{Exception UserExists}
	\label{tab:userExists}\\
	\hline%
	\textbf{Field} & \textbf{Type} & \textbf{Documentation} \\\hline
	    --- & --- & --- \\\hline
   \multicolumn{3}{|p{.95\linewidth}|}{\textbf{Used in:} Users.createUser} \\\hline
\end{longtable}
\subsubsection{UserNotFound}
The specified username does not exist.

\begin{longtable} {|p{4cm}|p{4cm}|p{7.25cm}|}
	\caption{Exception UserNotFound}
	\label{tab:userNotFound}\\
	\hline%
	\textbf{Field} & \textbf{Type} & \textbf{Documentation} \\\hline
	    --- & --- & --- \\\hline
   \multicolumn{3}{|p{.95\linewidth}|}{\textbf{Used in:} Users.updateProfile, Users.updatePassword, Users.deleteUser} \\\hline
\end{longtable}
\subsubsection{VCSAuthenticationFailed}
The client failed to prove its identity in the VCS.

\begin{longtable} {|p{4cm}|p{4cm}|p{7.25cm}|}
	\caption{Exception VCSAuthenticationFailed}
	\label{tab:vCSAuthenticationFailed}\\
	\hline%
	\textbf{Field} & \textbf{Type} & \textbf{Documentation} \\\hline
	    --- & --- & --- \\\hline
   \multicolumn{3}{|p{.95\linewidth}|}{\textbf{Used in:} Hawk.addRepository, OfflineCollaboration.checkout, OfflineCollaboration.commit, OfflineCollaboration.pull, OfflineCollaboration.publishLockDefinition, OfflineCollaboration.unpublishLockDefinition, OfflineCollaboration.lock, OfflineCollaboration.unlock} \\\hline
\end{longtable}
\subsubsection{VCSAuthorizationFailed}
The client does not have the required permissions in the VCS to perform the operation.

\begin{longtable} {|p{4cm}|p{4cm}|p{7.25cm}|}
	\caption{Exception VCSAuthorizationFailed}
	\label{tab:vCSAuthorizationFailed}\\
	\hline%
	\textbf{Field} & \textbf{Type} & \textbf{Documentation} \\\hline
	    --- & --- & --- \\\hline
   \multicolumn{3}{|p{.95\linewidth}|}{\textbf{Used in:} OfflineCollaboration.checkout, OfflineCollaboration.commit, OfflineCollaboration.pull, OfflineCollaboration.publishLockDefinition, OfflineCollaboration.unpublishLockDefinition, OfflineCollaboration.lock, OfflineCollaboration.unlock} \\\hline
\end{longtable}
